\section{Quasimodular forms} \label{quasimodular-forms}

This section introduces quasimodular forms as described in \cite{Kaneko-Zagier1995}.

\subsection{The space of modular forms}

Let $\mc{H}=\{\tau \in \Cbb \scl \Img(\tau)>0\}$ denote the upper half-plane. For $\tau \in \mc{H}$, define $q=\exp(2\pi i\tau)$ and $Y=4\pi\Img(\tau)$. Further, let $\SL_2(\Zbb) \subset \SL_2(\Cbb)$ denote the full modular group. Then $\SL_2(\Zbb)$ acts \if\footnotemark\fi on $\mc{H}$ by 
\[\gamma\tau = \frac{a\tau + b}{c\tau + d}\text{, for }\gamma = 
\begin{pmatrix}
 a & b \\
 c & d
\end{pmatrix}
\in \SL_2(\Zbb). 
\]

%\footnotetext{To see that $\gamma \tau \in \mc{H}$, note that $\Im(\gamma\tau)=\Im(\tau)/|c\tau+d|^2$.}

\begin{defi} Let $f\cl \mc{H} \to \Cbb$ be a function, let $k\in \Zbb$.

  1. The function $f$ is \emph{$\Zbb$-periodic}, if it satisfies $f(\tau + 1) = f(\tau)$ for all $\tau \in \mc{H}$. In this case there exists a  function $\tilde f\cl B\setminus \{0\} \to \Cbb$, defined on the open unit ball $B\subset \Cbb$ with the origin removed, such that $f(\tau)=\tilde f(q)$ for all $\tau$. Now let $f$ be holomorphic. Then so is $\tilde f$. We say that $f$ is \emph{holomorphic at infinity}, if $\tilde f$ has a holomorphic continuation to the whole of $B$.
  
  2. The function $f$ is said to satisfy the \emph{modular condition of weight $k$}, if \[f(\gamma\tau)=(c\tau+d)^kf(\tau)\] for all $\tau$ in $\mc{H}$ and all $\gamma \in \SL_2(\Zbb)$. Such a function is $\Zbb$-periodic, as can be seen by setting $\gamma = 
\bigl(\begin{smallmatrix}
 1 & 1 \\
 0 & 1
\end{smallmatrix}\bigr).$
  
  3. The function $f$ is a \emph{modular form (of weight $k$)} if it is holomorphic, satisfies the modular condition and is holomorphic at infinity.
\end{defi}

Note that if $k$ is odd, then any function satisfying the modular condition of weight $k$ is zero. This follows by using the modular condition with $\gamma = 
\bigl(\begin{smallmatrix}
 -1 & 0 \\
 0 & -1
\end{smallmatrix}\bigr).$
There are several alternate conventions for handling the weights $k$. Some authors for instance replace $k$ by $2k$ throughout, so that ``modular forms of weight $2k$'' are considered. This is the convention used by \cite{Serre1973}.

The modular forms of weight $k$ form a vector space, denoted\footnotemark by $\Mrm_k$. Multiplying two modular forms of weights $k$ respectively $l$ yields a modular form of weight $k+l$, giving the space $\bigoplus_k \Mrm_k$ the structure of a graded ring, denoted by $\Mrm_{*}$.

\footnotetext{
In \cite{Serre1973}, the space of modular forms of weight $2k$ is denoted by $\Mrm_k$.
}

\begin{expls} \label{ex:delta}
 For an even integer $k\geq 2$, the \emph{Eisenstein series\footnotemark of weight $k$} is the function \[E_k(\tau)=1-\frac{2k}{b_k}\sum_{n\geq 1}\sigma_{k-1}(n)q^n,\] where $b_k$ is the $k$-th Bernoulli number, and $\sigma_{k-1}(n)=\sum_{m|n}m^{k-1}$. By definition, these functions are holomorphic at infinity.
 
 For $k\geq 4$, the Eisenstein series of weight $k$ is a modular form of weight $k$. One proves this for example by showing that for $k\geq 4$, the series $E_k$ is a multiple of the function $G_k(\tau)=\sum_{(m,n)\in\Zbb^2\smallsetminus(0,0)}(m\tau+n)^{-k}$, which is indeed modular of weight $k$, see \cite[Ch.~VII, Prop.~8]{Serre1973} and \cite[Ch.~VII, 2.3]{Serre1973}.
 
 The function $\Delta=2^{-6}3^{-3}(E_4^3-E_6^2)$ is a modular form of weight $12$. By a theorem of Jacobi \cite[Ch.~VII, Thm.~6]{Serre1973}, one has \[\Delta(\tau)=q\prod_{n=1}^{\infty}(1-q)^{24}.\]
\end{expls}

\footnotetext{In \cite{Serre1973}, the Eisenstein series of weight $k$ as defined below is denoted by $E_{k/2}$. A similar remark applies to the function $G_k$ below.}

\begin{prop} \label{baby-dimension-formula}
 We have the following dimension formula:
 \[
  \dim(\Mrm_k(\SL_2(\Zbb)))=\begin{dcases}
                      \floor{{k}/{12}} & \text{ if $k\geq 0$ and $k\equiv 2 \pmod{12}$}\\
                      \floor{{k}/{12}}+1 & \text{ if $k\geq 0$ and $k\not\equiv 2 \pmod{12}$}\\
                      0 & \text{ if $k<0$}
                     \end{dcases}.
 \]
\end{prop}
\begin{proof}
 An elementary approach is given in \cite[Ch.~VII, Corollary 1]{Serre1973}. Alternatively, this formula arises as a corollary (\ref{dimension-formula}) to a more general formula proven in the Appendix, Theorem \ref{g-dimension-formula}.
\end{proof}

\begin{prop}
 There is an isomorphism of graded rings \[\Cbb[X_4, X_6]\xrightarrow{\sim} \Mrm_*\] mapping $X_i$ to $E_i$, where the former ring is graded by assigning to $X_i$ the degree $i$.
\end{prop}
\begin{proof}
 See \cite[Ch.~VII, Corollary 2]{Serre1973}.
\end{proof}

\subsection{The space of quasimodular forms}

Let $\mc{O}(\mc{H})$ denote the vector space of $\Cbb$-valued holomorphic functions on $\mc{H}$. Recall the imaginary part function $Y(\tau)=4\pi\Img(\tau)$. The following proposition shows that one may compare coefficients of elements of $\mc{O}(\mc{H})[Y^{-1}]$ as if $Y$ was a formal variable.

\begin{prop}
 Let $F=\sum_{m=0}^Mf_mY^{-m}$ be an element of $\mc{O}(\mc{H})[Y^{-1}]$. If $F=0$, then $f_m=0$ for all $m$.
\end{prop}
\begin{proof}
 For the differential operator $\frac{\dif}{\dif\conj{\tau}}$ one has $\frac{\dif}{\dif\conj{\tau}}Y^{-m}=-2\pi imY^{-m-1}$ and $\frac{\dif}{\dif\conj{\tau}}f_m=0$, hence \[0=\frac{\dif}{\dif\conj{\tau}}F(\tau)=-2\pi i\sum_{m=1}^{M}f_m(\tau)Y^{-m-1}=-2\pi iY^{-2}(\sum_{m=0}^{M-1}f_{m+1}{\tau}Y^{-m}).\]
 By induction this implies that the $f_m$ are zero for $m\geq 1$, hence also $f_0=0$.
\end{proof}

\begin{cor}
 Let $F=\sum_{m=0}^Mf_mY^{-m}$ be an element of $\mc{O}(\mc{H})[Y^{-1}]$ satisfying the modular condition of weight $k$. Then the $f_m$ are $\Zbb$-periodic.
\end{cor}

\begin{defi}
 An \emph{almost holomorphic modular form (of weight $k$)} is an element
 \[F=\sum_{m=0}^Mf_mY^{-m}\]
 of $\mc{O}(\mc{H})[Y^{-1}]$ such that $F$ satisfies the modular condition and the
 $f_m\cl \mc{H}\to \Cbb$ are holomorphic at infinity.
\end{defi}

\begin{prop}
 Let $F(\tau)=\sum_{m=0}^Mf_m(\tau)Y^{-m}$ be an almost holomorphic modular form. Then the leading coefficient $f_M$ is a modular form of weight $k-2M$. In particular, if $f_M \neq 0$, then $2M \leq k$.
\end{prop}
\begin{proof}
 This follows after comparing the coefficients of $Y^{-M}$ in both sides of the modularity condition $F(\gamma\tau)=(c\tau+d)^kF(\tau)$, using the equality \[Y^{-1}(\gamma\tau)=(c\tau+d)^2Y(\tau)^{-1}+\frac{c(c\tau+d)}{2\pi i}\]
 for $\gamma=
 \bigl(\begin{smallmatrix}
 a&b\\ c&d
 \end{smallmatrix} \bigr)
 \in \SL_n(\Zbb).$
\end{proof}

The almost holomorphic modular forms of weight $k$ form a vector space, denoted by $\widehat{\Mrm}_k$. Let $\widehat{\Mrm}_*$ denote the associated graded ring.

\begin{defi}
 An element in the image of the map $\widehat{\Mrm}_k\to\mc{O}(\mc{H})$ taking an almost holomorphic modular form $F=\sum_{m=0}^Mf_mY^{-m}$ of weight $k$ to $f_0$ is called a \emph{quasimodular form of weight $k$}. Hence a quasimodular form is a holomorphic function on the upper plane appearing as the constant term of an almost holomorphic modular form.
\end{defi}

Again, denote the vector space of quasimodular forms of weight $k$ by $\widetilde{\Mrm}_k$ and the associated graded ring by $\widetilde{\Mrm}_*$. The definition gives a surjective graded ring homomorphism $\widehat{\Mrm}_*\to\widetilde{\Mrm}_*$ and one has $\widehat{\Mrm}_k \cap \widetilde{\Mrm}_k = \Mrm_k$, the intersection being taken in the set of functions $\mc{H}\to\Cbb$.

\begin{expl}

Consider the second Eisenstein series \[E_2(\tau)=1-24\sum_{n\geq1}\sigma_1(n)q^n,\] where $\sigma_1(n)=\sum_{d|n}d$. For the weight $12$ modular form $\Delta(\tau)=q\prod_{n=1}^{\infty}(1-q)^{24}$, one has the identity $2\pi iE_2(\tau)=\frac{\dif}{\dif\tau}\log(\Delta(\tau))$, which is proven by a straightforward computation. Using the modularity of $\Delta$, one then computes \[E_2(\gamma\tau)=(c\tau+d)^2 E_2(\tau) + \frac{6c(c\tau+d)}{\pi i},\]

for $\gamma=
 \bigl(\begin{smallmatrix}
 a&b\\ c&d
 \end{smallmatrix} \bigr)
 \in \SL_n(\Zbb)$.

Now, since $Y^{-1}(\gamma\tau)=(c\tau+d)^2Y(\tau)^{-1}+\frac{c(c\tau+d)}{2\pi i}$, it follows that $E_2^*=E_2-12/Y$ is an almost holomorphic modular form of weight $2$. Hence, $E_2$ is a quasimodular form of weight $2$.

\end{expl}

\begin{prop} \label{pr:derivative} The space $\widetilde{\Mrm}_*$ of quasimodular forms satisfies the following properties.

  \itemrm{1.} The canonical graded homomorphism $\widehat{\Mrm}_* \to \widetilde{\Mrm}_*$ is an isomorphism.
  
  \itemrm{2.} There is an isomorphism of graded rings $\Mrm_* \otimes \Cbb[X_2] \simeq \Cbb[X_2, X_4, X_6]\to \widetilde{\Mrm}_*$ mapping $X_i$ to $E_i$, where the former ring is graded by assigning to $X_i$ the degree $i$.
  
  \itemrm{3.} Quasimodular forms are closed under taking derivatives. More precisely, the derivative of a quasimodular form of weight $k$ is a quasimodular form of weight $k$+2.
\end{prop}
\begin{proof}\ 

  1. The map $\widehat{\Mrm}_* \to \widetilde{\Mrm}_*$ is surjective by definition. Injectivity follows from Calculation \ref{cp:almost-holomorphic-modular-form-no-constant-term} below. Given an almost holomorphic modular form $F(\tau)=\sum_{m=1}^Mf_m(\tau)Y^{-m}$ with constant term zero, the strategy is to solve the modularity equation for the coefficients $f_m$. This way, one finds for a fixed argument $\tau$ a polynomial equation in the lower row components  $c,d$ of any transformation $\gamma \in \SL_2(\Zbb)$, involving the coefficients $f_m(\tau)$. By varying the transformation $\gamma$, one may force these coefficients to be zero.
  
  2. Express the map $\Cbb[X_2, X_4, X_6]\to \widetilde{\Mrm}_*$ as the composition \[\Cbb[X^*_2, X_4, X_6]\to \widehat{\Mrm}_* \to \widetilde{\Mrm}_*,\] where the first map takes $X^*_2$ to $E_2^*$ and $X_i$ to $E_i$, and the second map is the canonical map, which is an isomorphism by the first point above.
  
  To prove the surjectivity of the first map, let $F(\tau)=\sum_{m=0}^Mf_m(\tau)Y^{-m}$ be an almost holomorphic modular form. Then $f_M (E_2^*/12)^M$ is an almost holomorphic modular form of weight $k$, since $f_M$ is modular of weight $k-2M$, and the difference $F - f_M (E_2^*/12)^M$ has degree smaller than $M$. Now use induction on $M$.
  
  To get injectivity, let $F=\sum_{\alpha=0}^{k/2}(E_2^*)^\alpha f_{k-2\alpha}$ be an almost holomorphic modular form of weight $k$, in the image of the first map, where the $f_m$ are modular of weight $m$. If $F=0$, then by comparing the coefficients of $Y^{-k/2}$ one obtains $0=f_0$. Now it follows by induction on $k$ that the other coefficients $f_m$ are zero. Hence $F$ was the image of the zero element in $\Mrm_* \otimes \Cbb[X^*_2]$.
  
  3. To prove the last statement, one verifies that $(6/\pi i) E'_2-E_2^2$ is modular of weight $4$, and that if $f$ is modular of weight $k$, then $(6/\pi i)f' - kE_2f$ is modular of weight $2+k$. Now use the second point above.
\end{proof}

\begin{calc} \label{cp:almost-holomorphic-modular-form-no-constant-term}
 This calculation follows the one found in \cite{Bloch-Okounkov}.
 Let $F(\tau)=\sum_{m=1}^Mf_m(\tau)Y^{-m}$ be an almost holomorphic modular form, $\gamma=
 \bigl(\begin{smallmatrix}
 a&b\\ c&d
 \end{smallmatrix} \bigr)
 \in \SL_n(\Zbb)$,
 and $\tau \in \mc{H}$.
 Write $j=c\tau+d$, and $a=6cj/2\pi i$. Then $Y^{-1}(\gamma\tau)=a+j^2Y(\tau)^{-1}$. Hence,
 \begin{align*}
  F(\gamma\tau)&=\sum_{m=1}^Mf_m(\gamma\tau)(a+j^2Y^{-1})^m\\
               &=\sum_{m=1}^M\sum_{l=0}^m\binom{m}{l}f_m(\gamma\tau)a^{m-l}j^{2l}Y^{-l}\\
               &=\sum_{m=1}^Mf_m(\gamma\tau)a^m + \sum_{l=1}^M\sum_{m=l}^M\binom{m}{l}f_m(\gamma\tau)a^{m-l}j^{2l}y^{-l}.
 \end{align*}
 On the other hand,
 \[F(\gamma\tau)=\sum_{l=1}^Mf_l(\tau)j^kY^{-l},\]
 by the modularity condition. By comparing the coefficients of $Y^{-l}$, one obtains the equalities
 \begin{equation} \label{eq:coeff-zero}
 \sum_{m=1}^Mf_m(\gamma\tau)a^m=0
 \end{equation} and \[j^kf_l(\tau)=\sum_{m=l}^M\binom{m}{l}f_m(\gamma\tau)a^{m-l}j^{2l}.\]
 Rewriting the second equality yields
 \begin{equation} \label{eq:coeff-l}
  f_l(\gamma\tau)=f_l(\tau)j^{k-2l}-\sum_{m=l+1}^M\binom{m}{l}f_m(\gamma\tau)a^{m-l}.
 \end{equation}

 The latter may be solved recursively, starting by $f_M$, to get equalities of the form
 \begin{equation}
  f_l(\gamma\tau)=\text{(a polynomial in the $f_{\geq l}(\tau)$ , $j$ and $c$)}.
 \end{equation}
 The first two equalities are
 \begin{align*}
  f_M(\gamma\tau)&=f_M(\tau)j^{k-2M}\\
  f_{M-1}(\gamma\tau)&=f_{M-1}(\tau)j^{k-2M+2}-\text{const}\cdot f_M(\tau)j^{k-2M+1}c.
 \end{align*}
 In general, a straightforward inductive argument shows that in the summands of the expression \eqref{eq:coeff-l} for $f_l(\gamma\tau)$, the variable $j$ appears with a power lower than or equal to $k-2l$.
 Now let $r$ be the greatest index such that $f_r \neq 0$. Equation \eqref{eq:coeff-zero} finally gives, after substituting back the expressions for $j$ and $a$ and using \eqref{eq:coeff-l} for $l=r$, the relation
 \begin{align*}
 0&=\kappa_1 f_r(\gamma\tau)(c\tau+d)^rc^r+\sum_{l=r+1}^M\kappa_3 f_l(\gamma\tau)(c\tau+d)^lc^l\\
  &=\kappa_1 f_r(\tau)(c\tau+d)^{k-r}c^r- \\
  &- \sum_{m=r+1}^M\kappa_2\binom{m}{r}f_m(\gamma\tau)(c\tau+d)^{m-r}c^{m-r} + \sum_{l=r+1}^M\kappa_3 f_l(\gamma\tau)(c\tau+d)^lc^l,
 \end{align*}
 where the $\kappa_i$ are some nonzero constants.
 To obtain a contradiction, choose a point $\tau$ in the upper half-plane and consider the last relation as a polynomial equation in $c$ and $d$, letting $P(c,d)$ denote the right-hand side of the equation. First look for the possible coefficients of monomials of the form $c^rd^{\geq1}$. This excludes the third summand from the picture, since there $c$ will always appear with a power greater than $r$. Next look for the possible coefficients of the monomial $c^rd^{k-r}$. As seen when recursively solving the equations for $f_l(\gamma\tau)$, the second summand will include only terms where $(c\tau+d)$ appears with a power lower than $k-r$. Hence the coefficient of $c^rd^{k-r}$ in $P(c,d)$ is $\kappa_1f_r(\tau)$.
 
 Now, if $c\in \Zbb$, then there are infinitely many $d\in \Zbb$ such that $P(c,d)=0$. Indeed, there are infinitely many $d$ with $\gcd(c,d)=1$. For these $d$, find $a,b \in \Zbb$ such that $ad-bc=1$. Since
 $\bigl(\begin{smallmatrix}
 a&b\\ c&d
 \end{smallmatrix} \bigr)
 \in \SL_2(\Zbb)$,
 it follows that $P(c,d)=0$. Similarly, for all $d\in \Zbb$, there are infinitely many $c$ such that $P(c,d)=0$. It this follows that $P(c,d)=0$ holds for all $c,d\in \Cbb$. These remarks may be summarized by the statement that the set of all $c,d$ belonging to the lower row of some matrix in $\SL_2(\Zbb)$ is Zariski-dense in $\Cbb^2$.
 
 Concluding, since $P$ is zero as a function on $\Cbb^2$, it is also zero as a polynomial, hence the coefficient $\kappa_1f_r(\tau)$ is zero. Since $\tau$ was arbitrary, one finds $f_r=0$, a contradiction.
\end{calc}

%\subsection{A criterium for being quasimodular}

