\section{Quasimodular forms}

This section introduces quasimodular forms as described in \cite{Kaneko-Zagier1995}.

\subsection{The space of modular forms}

Let $\mc{H}=\{\tau \in \Cbb \scl \Img(\tau)>0\}$ denote the upper half-plane. For $\tau \in \mc{H}$, define $q=\exp(2\pi i\tau)$ and $Y=4\pi\Img(\tau)$. Further, let $\SL_2(\Zbb) \subset \SL_2(\Cbb)$ denote the full modular group. Then $\SL_2(\Zbb)$ acts \if\footnotemark\fi on $\mc{H}$ by 
\[\gamma\tau = \frac{a\tau + b}{c\tau + d}\text{, for }\gamma = 
\begin{pmatrix}
 a & b \\
 c & d
\end{pmatrix}
\in \SL_2(\Zbb). 
\]

%\footnotetext{To see that $\gamma \tau \in \mc{H}$, note that $\Im(\gamma\tau)=\Im(\tau)/|c\tau+d|^2$.}

\begin{defi} Let $f\cl \mc{H} \to \Cbb$ be a function, let $k\in \Zbb$.
 \begin{enumerate} 
  \item The function $f$ is \emph{$\Zbb$-periodic}, if it satisfies $f(\tau + 1) = f(\tau)$ for all $\tau \in \mc{H}$. In this case there exists a  function $\tilde f\cl B\setminus \{0\} \to \Cbb$, defined on the open unit ball $B\subset \Cbb$ with the origin removed, such that $f(\tau)=\tilde f(q)$ for all $\tau$. Now let $f$ be holomorphic. Then so is $\tilde f$. We say that $f$ is \emph{holomorphic at infinity}, if $\tilde f$ has a holomorphic continuation to the whole of $B$.
  
  \item The function $f$ is said to satisfy the \emph{modular condition of weight $k$}, if \[f(\gamma\tau)=(c\tau+d)^kf(\tau)\] for all $\tau$ in $\mc{H}$ and all $\gamma \in \SL_2(\Zbb)$. Such a function is $\Zbb$-periodic, as can be seen by setting $\gamma = 
\bigl(\begin{smallmatrix}
 1 & 1 \\
 0 & 1
\end{smallmatrix}\bigr).$
  
  \item The function $f$ is a \emph{modular form (of weight $k$)} if it is holomorphic, satisfies the modular condition and is holomorphic at infinity.
 \end{enumerate}
\end{defi}

Note that if $k$ is odd, then any function satisfying the modular condition of weight $k$ is zero. This follows by using the modular condition with $\gamma = 
\bigl(\begin{smallmatrix}
 -1 & 0 \\
 0 & -1
\end{smallmatrix}\bigr).$
There are several alternate conventions for handling the weights $k$. Some authors for instance replace $k$ by $2k$ throughout, so that ``modular forms of weight $2k$'' are considered. This is the convention used by \cite{Serre1973}.

The modular forms of weight $k$ form a vector space, denoted\footnotemark by $\M_k$. Multiplying two modular forms of weights $k$ respectively $l$ yields a modular form of weight $k+l$, giving the space $\bigoplus_k \M_k$ the structure of a graded ring, denoted by $\M_{*}$.

\footnotetext{
In \cite{Serre1973}, the space of modular forms of weight $2k$ is denoted by $\M_k$.
}

\begin{expls} \label{ex:delta}
 For an even integer $k\geq 2$, the \emph{Eisenstein series\footnotemark of weight $k$} is the function \[E_k(\tau)=1-\frac{2k}{b_k}\sum_{n\geq 1}\sigma_{k-1}(n)q^n,\] where $b_k$ is the $k$-th Bernoulli number, and $\sigma_{k-1}(n)=\sum_{m|n}m^{k-1}$. By definition, these functions are holomorphic at infinity.
 
 For $k\geq 4$, the Eisenstein series of weight $k$ is a modular form of weight $k$. One proves this for example by showing that for $k\geq 4$, the series $E_k$ is a multiple of the function $G_k(\tau)=\sum_{(m,n)\in\Zbb^2\smallsetminus(0,0)}(m\tau+n)^{-k}$, which is indeed modular of weight $k$, see \cite[Ch.~VII, Prop.~8]{Serre1973} and \cite[Ch.~VII, 2.3]{Serre1973}.
 
 The function $\Delta=2^{-6}3^{-3}(E_4^3-E_6^2)$ is a modular form of weight $12$. By a theorem of Jacobi \cite[Ch.~VII, Thm.~6]{Serre1973}, one has \[\Delta(\tau)=q\prod_{n=1}^{\infty}(1-q)^{24}.\]
\end{expls}

\footnotetext{In \cite{Serre1973}, the Eisenstein series of weight $k$ as defined below is denoted by $E_{k/2}$. A similar remark applies to the function $G_k$ below.}

\begin{prop}
 There is an isomorphism of graded rings \[\Cbb[X_4, X_6]\xrightarrow{\sim} \M_*\] mapping $X_i$ to $E_i$, where the former ring is graded by assigning to $X_i$ the degree $i$. In particular, there are no nonzero modular forms of negative weight.
\end{prop}
\begin{proof}
 See \cite[Ch.~VII, 3.1, 3.2]{Serre1973}
\end{proof}


\subsection{The space of quasimodular forms}

Let $\mc{O}(\mc{H})$ denote the vector space of $\Cbb$-valued holomorphic functions on $\mc{H}$. Recall the imaginary part function $Y(\tau)=4\pi\Img(\tau)$. The following proposition shows that one may compare coefficients of elements of $\mc{O}(\mc{H})[Y^{-1}]$ as if $Y$ was a formal variable.

\begin{prop}
 Let $F=\sum_{m=0}^Mf_mY^{-m}$ be an element of $\mc{O}(\mc{H})[Y^{-1}]$. If $F=0$, then $f_m=0$ for all $m$.
\end{prop}
\begin{proof}
 For the differential operator $\frac{\dif}{\dif\conj{\tau}}$ one has $\frac{\dif}{\dif\conj{\tau}}Y^{-m}=-2\pi imY^{-m-1}$ and $\frac{\dif}{\dif\conj{\tau}}f_m=0$, hence \[0=\frac{\dif}{\dif\conj{\tau}}F(\tau)=-2\pi i\sum_{m=1}^{M}f_m(\tau)Y^{-m-1}=-2\pi iY^{-2}(\sum_{m=0}^{M-1}f_{m+1}{\tau}Y^{-m}).\]
 By induction this implies that the $f_m$ are zero for $m\geq 1$, hence also $f_0=0$.
\end{proof}

\begin{cor}
 Let $F=\sum_{m=0}^Mf_mY^{-m}$ be an element of $\mc{O}(\mc{H})[Y^{-1}]$ satisfying the modular condition of weight $k$. Then the $f_m$ are $\Zbb$-periodic.
\end{cor}

\begin{defi}
 An \emph{almost holomorphic modular form (of weight $k$)} is an element
 \[F=\sum_{m=0}^Mf_mY^{-m}\]
 of $\mc{O}(\mc{H})[Y^{-1}]$ such that $F$ satisfies the modular condition and the
 $f_m\cl \mc{H}\to \Cbb$ are holomorphic at infinity.
\end{defi}

\begin{prop}
 Let $F(\tau)=\sum_{m=0}^Mf_m(\tau)Y^{-m}$ be an almost holomorphic modular form. Then the leading coefficient $f_M$ is a modular form of weight $k-2M$. In particular, if $f_M \neq 0$, then $2M \leq k$.
\end{prop}
\begin{proof}
 This follows after comparing the coefficients of $Y^{-M}$ in both sides of the modularity condition $F(\gamma\tau)=(c\tau+d)^kF(\tau)$, using the equality \[Y^{-1}(\gamma\tau)=(c\tau+d)^2Y(\tau)^{-1}+\frac{c(c\tau+d)}{2\pi i}\]
 for $\gamma=
 \bigl(\begin{smallmatrix}
 a&b\\ c&d
 \end{smallmatrix} \bigr)
 \in \SL_n(\Zbb).$
\end{proof}

The almost holomorphic modular forms of weight $k$ form a vector space, denoted by $\widehat{\M}_k$. Let $\widehat{\M}_*$ denote the associated graded ring.

\begin{defi}
 An element in the image of the map $\widehat{\M}_k\to\mc{O}(\mc{H})$ taking an almost holomorphic modular form $F=\sum_{m=0}^Mf_mY^{-m}$ of weight $k$ to $f_0$ is called a \emph{quasimodular form of weight $k$}. Hence a quasimodular form is a holomorphic function on the upper plane appearing as the constant term of an almost holomorphic modular form.
\end{defi}

Again, denote the vector space of quasimodular forms of weight $k$ by $\widetilde{\M}_k$ and the associated graded ring by $\widetilde{\M}_*$. The definition gives a surjective graded ring homomorphism $\widehat{\M}_*\to\widetilde{\M}_*$ and one has $\widehat{\M}_k \cap \widetilde{\M}_k = \M_k$.

\begin{expl}

Consider the second Eisenstein series \[E_2(\tau)=1-24\sum_{n\geq1}\sigma_1(n)q^n,\] where $\sigma_1(n)=\sum_{d|n}d$. For the weight $12$ modular form $\Delta(\tau)=q\prod_{n=1}^{\infty}(1-q)^{24}$, one has the identity $2\pi iE_2(\tau)=\frac{\dif}{\dif\tau}\log(\Delta(\tau))$, which is proven by a straightforward computation. Using the modularity of $\Delta$, one then computes \[E_2(\gamma\tau)=(c\tau+d)^2 E_2(\tau) + \frac{6c(c\tau+d)}{\pi i},\]

for $\gamma=
 \bigl(\begin{smallmatrix}
 a&b\\ c&d
 \end{smallmatrix} \bigr)
 \in \SL_n(\Zbb)$.

Now, since $Y^{-1}(\gamma\tau)=(c\tau+d)^2Y(\tau)^{-1}+\frac{c(c\tau+d)}{2\pi i}$, it follows that $E_2^*=E_2-12/Y$ is an almost holomorphic modular form of weight $2$. Hence, $E_2$ is a quasimodular form of weight $2$.

\end{expl}

\begin{prop} The space $\widetilde{\M}_*$ of quasimodular forms satisfies the following properties.
 \begin{enumerate}
  \item The canonical graded homomorphism $\widehat{\M}_* \to \widetilde{\M}_*$ is an isomorphism.
  \item There is an isomorphism of graded rings $\M_* \otimes \Cbb[X_2] \simeq \Cbb[X_2, X_4, X_6]\to \widetilde{\M}_*$ mapping $X_i$ to $E_i$, where the former ring is graded by assigning to $X_i$ the degree $i$.
  \item Quasimodular forms are closed under taking derivatives.
 \end{enumerate}
\end{prop}
\begin{proof}
 \begin{enumerate}
  \item The map $\widehat{\M}_* \to \widetilde{\M}_*$ is surjective by definition. Injectivity follows from Calculation \ref{cp:almost-holomorphic-modular-form-no-constant-term}. Given an almost holomorphic modular form $F(\tau)=\sum_{m=1}^Mf_m(\tau)Y^{-m}$ with constant term zero, the strategy is to solve the modularity equation for the coefficients $f_m$. This way, one finds for a fixed argument $\tau$ a polynomial equation in the lower row components  $c,d$ of any transformation $\gamma \in \SL_2(\Zbb)$, involving the coefficients $f_m(\tau)$. By varying the transformation $\gamma$, one may force these coefficients to be zero.
  \item Express the map $\Cbb[X_2, X_4, X_6]\to \widetilde{\M}_*$ as the composition \[\Cbb[X^*_2, X_4, X_6]\to \widehat{\M}_* \to \widetilde{\M}_*,\] where the first map takes $X^*_2$ to $E_2^*$ and $X_i$ to $E_i$, and the second map is the canonical map, which is an isomorphism by the first point above.
  
  To prove the surjectivity of the first map, let $F(\tau)=\sum_{m=0}^Mf_m(\tau)Y^{-m}$ be an almost holomorphic modular form. Then $f_M (E_2^*/12)^M$ is an almost holomorphic modular form of weight $k$, since $f_M$ is modular of weight $k-2M$, and the difference $F - f_M (E_2^*/12)^M$ has degree smaller than $M$. Now use induction on $M$.
  
  To get injectivity, let $F=\sum_{\alpha=0}^{k/2}(E_2^*)^\alpha f_{k-2\alpha}$ be an almost holomorphic modular form of weight $k$, in the image of the first map, where the $f_m$ are modular of weight $m$. If $F=0$, then by comparing the coefficients of $Y^{-k/2}$ one obtains $0=f_0$. Now it follows by induction on $k$ that the other coefficients $f_m$ are zero. Hence $F$ was the image of the zero element in $\M_* \otimes \Cbb[X^*_2]$.
  \item To prove the last statement, one verifies that $(6/\pi i) E'_2-E_2^2$ is modular of weight $4$, and that if $f$ is modular of weight $k$, then $(6/\pi i)f' - kE_2f$ is modular of weight $2+k$. Now use the second point above.
 \end{enumerate}

\end{proof}

%\subsection{A criterium for being quasimodular}

