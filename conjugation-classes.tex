\section{Conjugacy classes of the symmetric group}

In this section, we further the computation of $\widehat{N}_{g,d}$ by using similar techniques to the one applied when counting cycles in a graph. The rough picture is one of a graph with vertices the conjugacy classes of $\Sym_d$ and edges representing the passage from one class to another by multiplication with a simple transposition. We seek to count not cycles, but cycles starting and ending with the same representative, in the sense specified in the section. To do this, we make use of an analogon of the adjacency matrix for a graph.

To abbreviate, we use the term ``transposition'' for simple transpositions.

\subsection{Conjugacy cycles}

Recall the definition
\begin{align*}
 \widehat{T}_{g,d} = \{(\tau_1,\dotsc,\tau_{2g-2},\sigma_1,\sigma_2)\in \Sym_d^{2g} \scl &\text{each }\tau_i\text{ is a transposition,}\\ &\tau_1\dotsm\tau_{2g-2}=\sigma_1\sigma_2\sigma_1^{-1}\sigma_2^{-1}\}.
\end{align*}
Our aim is now to rewrite this definition using conjugacy classes. Note that the condition in the definition is equivalent to
\begin{equation} \label{eq:conj-cond}
 (\tau_1 \dotsm \tau_{2g-2})\sigma_2=\sigma_1\sigma_2\sigma_1^{-1}.
\end{equation}

\begin{defi}
 For $\sigma_2 \in \Sym_d$, define
 \begin{align*}
  P_{g,d}(\sigma_2) = \{(\tau_1,\dotsc,\tau_{2g-2})\in \Sym_d^{2g-2} \scl &\text{each }\tau_i\text{ is a transposition,}\\ &\tau_1\dotsm\tau_{2g-2}\sigma_2\text{ is conjugate to }\sigma_2\}.
 \end{align*}
 If $g=1$, define $P_{g,d}$ to be the singleton set $\{\bullet\}$.
 Further, let $c(\sigma_2)$ denote the conjugacy class of $\sigma_2$.
\end{defi}

\begin{prop}
 Let $\mc{R}=(\sigma_2^{(1)},\dotsc,\sigma_2^{(r)})$ be a system of (distinct) representatives of the conjugacy classes of $\Sym_d$. Then \[\abs{\widehat{T}_{g,d}}=\sum_{\sigma_2\in\mc{R}}d!\abs{P_{g,d}(\sigma_2)}.\]
\end{prop}
\begin{proof}
 Let $\sigma_2 \in \Sym_d$, let $(\tau_1,\dotsc,\tau_{2g-2}) \in P_{g,d}(\sigma_2)$ and let $\sigma\pr_1$ be an element such that $(\tau_1 \dotsm \tau_{2g-2})\sigma_2=\sigma\pr_1\sigma_2(\sigma\pr_1)^{-1}$. Then there is a bijection of the set of elements $\sigma_1$ satisfying \eqref{eq:conj-cond} onto the set of elements commuting with $\sigma_2$, given by sending $\sigma_1$ to $(\sigma\pr_1)^{-1}\sigma_1$. The number of elements commuting with $\sigma_2$ is given by the cardinality of the stabilizer $\abs{\Stab(\sigma_2)}=\abs{\Sym_d}/\abs{c(\sigma_2)}=d!/\abs{c(\sigma_2)}$. Thus, one obtains 
 \[
  \abs{\widehat{T}_{g,d}}=\sum_{\sigma\in\Sym_d}\frac{d!}{\abs{c(\sigma)}}\abs{P_{g,d}(\sigma)}.
 \]
Further, the function $\abs{P_{g,d}}:\Sym_d\to\Cbb$ is constant on conjugacy classes. Indeed, for $\sigma\in\Sym_d$ there is a bijection of $P_{g,d}(\sigma_2)$ onto $P_{g,d}(\sigma\sigma_2\sigma^{-1})$ given by conjugation with $\sigma$ in each component. From this follows the required equality.
\end{proof}

\begin{cor} \label{prop:first-second-reduction-step}
 The above proposition, together with Lemma \ref{prop:first-reduction-step}, give the equality
 \[
  \widehat{N}_{g,d}=\sum_{\sigma_2\in\mathcal{R}}\abs{P_{g,c}(\sigma_2)}.
 \]
\end{cor}

From now on, let $\mc{R}=(\sigma_2^{(1)},\dotsc,\sigma_2^{(r)})$ be a fixed system of representatives of the conjugacy classes of $\Sym_d$. Then the cardinality $r=\partit(d)$ of $\mc{R}$ is the number of (unordered) partitions of $\{1,\dotsc,d\}$. This follows essentially from the fact that conjugation with a permutation acts on cycles by applying the permutation to the entries of the cycle.

\subsection{Adjacency matrices}

\begin{defi}
 Let $d\geq 1$ and $k\geq 0$.
 \begin{enumerate}
  \item For $1\leq i,j\leq r$, define the sets $N_{d,i,j}^k$ by
  \begin{align*}
   N_{d,i,j}^k = \{(\tau_1,\dotsc,\tau_k)\in \Sym_d^{k} \scl &\text{each }\tau_i\text{ is a transposition,}\\ &\tau_1\dotsm\tau_k\sigma_2^{(i)}\in c(\sigma_2^{(j)})\}.
  \end{align*}
  For $k=0$, define $N_{d,i,j}^0=\delta_{i,j}$ (Kronecker delta).
  \item Define the size $r$ square matrix $M_d$ by
  \[
   (M_d)_{i,j}=\abs{N_{d,i,j}^1}.
  \]
  This does not depend on the choice of system of representatives $\mc{R}$.
 \end{enumerate}
\end{defi}

\begin{rmk}
 If $k$ is odd, applying the signum homomorphism to the defining condition shows that $N_{d,i,i}^k$ is empty. If $k=2g-2$ is even, then $N_{d,i,i}^{2g-2}=P_{g,d}(\sigma_2^{(i)})$.
\end{rmk}

\begin{prop}
 The entries of $M_d^k$ are given by $(M_d^k)_{i,j}=\abs{N_{d,i,j}^k}$.
\end{prop}
\begin{proof}
 The proof is by induction on $k$. For $k=0,1$, there is nothing to show. For the induction step, note that if $i$ (resp. $j$) are fixed, the sets $N_{d,i,j}^k$ are pairwise disjoint for varying $j$ (reps. $i$). Now define a function
 \[
  \coprod_{l=1}^r N_{d,i,l}^k \times N_{d,l,j}^1 \to N_{d,i,j}^{k+1}
 \]
 as follows: for a given element $((\tau_1,\dotsc,\tau_k),\tau_0)$, let $\sigma\in\Sym_d$ be the unique element such that $\tau_1\dotsm\tau_k\sigma_2^{(i)}=\sigma\sigma_2^{(l)}\sigma^{-1}$, and define the image of $((\tau_1,\dotsc,\tau_k),\tau_0)$ to be $(\sigma\tau_0\sigma^{-1},\tau_1,\dotsc,\tau_k)$. By the definition of matrix multiplication, it suffices to prove that this function is a bijection.
 
 Injectivity is clear by the uniqueness of $\sigma$ in the definition. For surjectivity, given an element $(\tau^{\textrm{,}}_0,\tau_1,\dotsc,\tau_k)$ in the target, choose an $l$ such that $\tau_1\dotsm\tau_k\sigma_2^{(i)}$ is conjugate to $\sigma_2^{(l)}$, say $\tau_1\dotsm\tau_k\sigma_2^{(i)}=\sigma\sigma_2^{(l)}\sigma^{-1}$. Then $(\sigma^{-1}\tau_0\sigma)\sigma_2^{(l)}$ is conjugate to $\sigma_2^{(j)}$.
\end{proof}

\begin{lemma} \label{prop:counting-fct}
 Let $d\geq 1$ and $r=\partit(d)$. Let $\mu_{1,d},\dotsc,\mu_{r,d}$ be the eigenvalues of $M_d$, listed according to their algebraic multiciplicities. Then
 \[
  \widehat{Z}(q,\lambda)=\sum_{d\geq1}\sum_{i=1}^r \exp(\mu_{i,d}\lambda)q^d.
 \]
\end{lemma}
\begin{proof}
 Recall the definition of $\widehat{Z}$:
 \[\widehat{Z}(q,\lam) = \sum_{g\geq 1}\sum_{d\geq 1} \frac{\widehat{N}_{g,d}}{(2g-2)!}q^d\lam^{2g-2}.\]
 The above proposition and remark give $(M_d^{2g-2})_{i,i}=\abs{P_{g,d}(\sigma_2^{(i)})}$ and $(M_d^k)_{i,i}=0$ if $k$ is odd, for all $i$. Hence, by \ref{prop:first-second-reduction-step} one has $\widehat{N}_{g,d}=\Tr(M_d^{2g-2})=\sum_{i=1}^r \mu_{i,d}^{2g-2}$, and since the terms for $k$ odd vanish,
 \begin{align*}
  \widehat{Z}(q,\lam)&=\sum_{g\geq1}\sum_{d\geq1}\frac{\Tr(M^{2g-2}_d)}{(2g-2)!)}q^d\lam^{2g-2}\\
  &=\sum_{d\geq1}\sum_{i=1}^r\sum_{g\geq1}\frac{\mu_{i,d}^{2g-2}}{(2g-2)!}\lam^{2g-2}q^d\\
  &=\sum_{d\geq1}\sum_{i=1}^r \exp(\mu_{i,d}\lambda)q^d.
 \end{align*}
\end{proof}