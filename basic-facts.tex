\section{Basic facts and definitions}

This section collects some basic facts and definitions that will be of use later in this work.

\iffalse
\subsection{Algebraic Curves}


\begin{defi}
 The \emph{$n$-dimensional affine space} is the set $\Abb^n$ of $n$-tuples with entries in $\Cbb$.
\end{defi}

\begin{defi}
 For an ideal $J\subset \Cbb[x_1,\ldots,x_n]$ we define the \emph{zero set $V(J)$ of $J$} as the set of points $p \in \Cbb$. For a subset 
$X\subset \Abb^n$ we define the \emph{ideal $I(V)$ of $X$} as usual.
\end{defi}
\fi

\subsection{Covering spaces} \label{sec:covering-spaces}

\begin{defi}
 Let $X$ be a topological space, $F$ a set endowed with the discrete topology, and $G$ a group acting on both $X$ and $F$. Define the fibred product $X \times_{G} F$ to be the topological space $(X\times F)\; /\sim$, where $(x,f)\sim (gx,gf)$ for all $g$ in $G$.
\end{defi}

\if 0
\begin{joke} \ 
 \begin{enumerate}
 \item [Q: ] How do you call a team of actors in a Hollywood studio?
 \item [A: ] A group acting on a set.
 \end{enumerate}
\end{joke}
\fi

\begin{prop}

 Let $X$ be a connected, locally pathwise connected, and semi-locally simply connected topological space. Let $p\cl\widetilde{X}\to X$ be a universal cover. Furthermore, choose a point $\widetilde{x}_0$ of $\widetilde{X}$, and let $x_0$ be the image of $\widetilde{x}_0$ in $X$. Denote the fundamental group $\pi_1(X,x_0)$ by $\pi_1$. Then there is an eqivalence of categories
 \[
 \begin{tikzcd}
  \{\text{Unbranched covers of X}\} \arrow{r} & \{\pi_1\text{-sets}\},
 \end{tikzcd}
 \]
 defined by the pair of quasi-inverse functors \[(p_Y\cl Y\to X) \mapsto p^{-1}_Y(x_0)\;\text{ and }\;F \mapsto \widetilde{X} \times_{\pi_1} F.\]
\end{prop}
\begin{proof}
 One verifies by hand that the given functors are mutually quasi-inverse, by using elementary covering theory. Nonetheless, the needed isomorphisms between objects are given below.
 % Denote the group action of $\pi_1$ on $\widetilde X$ by $(\gamma,x)\mapsto \gamma\ldot x$.
 
 Let $F$ be a $\pi_1$-set and $p_F\cl \widetilde{X} \times_{\pi_1} F \to X$ the associated covering. Define a map $\zeta_F\cl F\to p_F^{-1}(x_0)$ by sending an element $f$ to the class of $(\widetilde{x}_0,f)$.
 % This map is surjective by definition, and is injective since the $\pi_1$-action on $\widetilde{X}$ is free.
 
 On the other hand, let $p_Y\cl Y\to X$ be a cover of $X$. Define a map \[\eta_Y\cl \widetilde{X}\times_{\pi_1} p_Y^{-1} \to Y\] as follows. For a given class $(\widetilde{x},f)$, let $\beta\cl [0,1] \to \widetilde{X}$ be a path starting in $\widetilde{x}_0$ and ending in $\widetilde{x}$. Consider the projection $p\beta$ of $\beta$ to $X$ and lift the path $p\beta$ to a path $\widetilde{\beta}_f$ in $Y$, with starting point $f$. Finally, set $\eta_Y(\widetilde{x},f)=\widetilde{\beta}_f(1)$. Note that since $\widetilde{X}$ is simply connected, this is independent of the choice of the path $\beta$. Also, the map is well-defined, since $p\beta \widetilde{\gamma} = p\beta$ for any lift $\widetilde{\gamma}$ of a loop in $X$.
 \if 0
 $\eta_Y$ is surjective: for $y \in Y$, let $\beta$ be a path in $X$ with starting point $p_Y(y)$ and endpoint $x_0$. Let $f=\widetilde{\beta}_{y}(1)$, be the endpoint of the lift of $\beta$ to $Y$ with starting point $y$. Then $y$ is the image of $(\widetilde{\beta}_{\widetilde{x}_0}(1),f)$ under $\eta_Y$, where $\widetilde{\beta}_{\widetilde{x}_0}$ is a lift of $\beta$ to $\widetilde{X}$ with starting point $\widetilde{x}_0$. To see that the map is injective, given any two points $(\widetilde{x}_1,f),(\widetilde{x}_2,g)$ mapping to the same point in $Y$, define a path in $\widetilde{X}$ connecting $\widetilde{x}_1$ to $\widetilde{x}_2$, and use the paths given by the definition of $\eta_Y$ to construct the loop in $X$ that will take $(\widetilde{x}_1,f)$ to $(\widetilde{x}_2,g)$
 \fi
\end{proof}

\begin{rmk}
 In the above proposition, if $X$ has the structure of a Riemann surface, then the first category may be taken to be the category of unbranched covers of Riemann surfaces over $X$. Indeed, every cover inherits a complex structure from $X$ such that the structure map becomes holomorphic, and morphisms of covers of $X$ are automatically holomorphic. Indeed, if $g\cl C'\to C$ is a continuous map and $f\cl C\to X$ is an open and holomorphic map such that $f\circ g$ is holomorphic, then $g$ is holomorphic; see \cite[1.3.7]{Lamotke2005}.
 
 Furthermore, let $X$ be a Riemann surface, let $S\subset X$ be a finite set. Then putting $(C,p)\mapsto (C\ssm p^{-1}(S),p)$ defines an equivalence of categories between the category of finite covers of $X$ with ramification locus contained in $S$ and the category of finite unbranched covers of $X\ssm S$. The reason is roughly that the local data of an unbranched cover around a ``missing'' branch point uniquely characterizes that of any extention of that cover to a ramified one, \eg the local degree of the cover map will correspond to the ramification index. The topic of extending unbranched covers to branched ones is discussed in detail in \cite[4.6]{Lamotke2005}.
\end{rmk}


\subsection{Complex curves}

\begin{prop} \label{prop:curves-to-fields}
 The assignment $C\mapsto K(C)$ defines a contravariant equivalence of categories between the category of irreducible smooth curves over $\Cbb$ 
and the 
category of finitely generated field extensions of $\Cbb$ of trascendence degree one. By definition, degree $d$ maps of curves correspond 
to degree $d$ field extensions.
\end{prop}

\begin{proof}
 See \cite[pp.20-22]{Silverman2009}
\end{proof}

\begin{prop}[Riemann--Hurwitz formula] \label{prop:hurwitz}
 Let $\varphi \cl C_1 \to C_2$ be a finite, degree $d$ map of smooth curves of genera $g_1$ and $g_2$, respectively. Then \[2g_1-2 = d(2g_2-2) + 
\sum_{x\in C_1}(\e_{\varphi}(x)-1),\] where $\e_\varphi(x)$ is the ramification index of $\varphi$ at $x$.
\end{prop}

\begin{proof}
 See \cite[Thm.~5.9]{Silverman2009} or \cite[7.2.1]{Lamotke2005}.
\end{proof}


\subsection{Further definitions}

\begin{defi}
 Let $X$ be a set. A \emph{weighting} on $X$ is a function $w\cl X\to[0,\infty]$. For an element $x$ of $X$, the value $w(x)$ is called the \emph{weight} of $x$. The \emph{weighted count of the elements of $X$} is defined as the sum $\sum_{x\in X}w(x)$. 
\end{defi}
