\section{Basic facts and definitions}

In this section we will fix some notation and recall the definitions and basic properties of the objects of this thesis. We will follow 
\cite{Silverman2009}.

\iffalse
\subsection{Algebraic Curves}


\begin{defi}
 The \emph{$n$-dimensional affine space} is the set $\Abb^n$ of $n$-tuples with entries in $\Cbb$.
\end{defi}

\begin{defi}
 For an ideal $J\subset \Cbb[x_1,\ldots,x_n]$ we define the \emph{zero set $V(J)$ of $J$} as the set of points $p \in \Cbb$. For a subset 
$X\subset \Abb^n$ we define the \emph{ideal $I(V)$ of $X$} as usual.
\end{defi}
\fi

\subsection{Complex curves}

\begin{prop} \label{prop:curves-to-fields}
 The assignment $C\mapsto K(C)$ defines a contravariant equivalence of categories between the category of irreducible smooth curves over $\Cbb$ 
and the 
category of finitely generated, transcendence degree one, field extensions of $\Cbb$. By definition, degree $d$ maps of curves correspond 
to degree $d$ field extensions.
\end{prop}

\begin{proof}
 See \cite{Silverman2009} pp. 20-22.
\end{proof}

\begin{prop}[Riemann-Hurwitz formula] \label{prop:hurwitz}
 Let $\varphi \cl C_1 \to C_2$ be a finite, degree $d$ map of smooth curves of genera $g_1$ and $g_2$, respectively. Then \[2g_1-2 = d(2g_2-2) + 
\sum_{x\in C_1}(\e_{\varphi}(x)-1),\] where $\e_\varphi(x)$ is the ramification index of $\varphi$ at $x$.
\end{prop}

