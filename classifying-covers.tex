\section{Classifying covers via the fundamental group}

Let $E$ be an elliptic curve, $S=\{b_1,\dotsc,b_{2g-2}\}$ a set of $2g-2$ distinct points of $E$. Fix a basis point $b_0\in E\smallsetminus S$, and denote the fundamental group $\pi_1(E\smallsetminus S, b_0)$ by $\pi_1$.
Recall the equivalence of categories from \ref{sec:covering-spaces}:

\[
 \begin{tikzcd}
  \{\text{Unbranched covers } of E\smallsetminus S\} \arrow{r} & \{\pi_1\text{-sets}\}.
 \end{tikzcd}
\]

The goal of this section is to use this equivalence of categories to classify those $\pi_1$-sets giving rise to unbranched covers that, after filling adding the branched points, become the covers we are interested in, i.e. the over $S$ simply branched, genus $g$, degree $d$ covers. To obtain natural $\pi_1$-actions on the set of $d$ fibre points of $b_0$, it is convenient to introduce markings on the set of fibres.

\subsection{Marked covers}

\begin{defi}
 A \emph{marked (degree $d$, genus $g$, simply branched over $S$) cover} of $E$ is a triple $(C,p,m)$, where $(C,p)\in \Cov(E,S)_{g,d}$ and $m\cl p^{-1}(b_0)\to \{1,\dotsc,d\}$ is a bijective map, the \emph{marking} of $(C,p,m)$.
 
 Two marked covers $(C_1,p_1,m_1)$ and $(C_2,p_2,m_2)$ are considered equivalent, if there is an isomorphism of covers $\phi\cl C_1 \to C_2$ such that $m_1=m_2\phi$. Let $\widetilde{\Cov}(E,S)_{g,d}$ denote the set of equivalence classes of marked covers with respect to this relation.
\end{defi}

\begin{defi}
 Let $(C,p)$ be a cover of $E$. Denote the group operation of $\pi_1$ on the fibre of $p^{-1}(b_0)$ by $(\gamma,x)\mapsto \gamma\ldot x$. Define the monodromy map \[\text{mon}\cl \widetilde{\Cov}(E,S)_{g,d} \to \Hom(\pi_1, S_d)\]
 by $\text{mon}(C,p,m)(\gamma)(i)=m(\gamma\ldot m^{-1}(i))$.
\end{defi}

Let the symmetric group $S_d$ operate on the first set by $\sigma \ldot (C,p,m)=(C,p,\sigma m)$, and on the second by $\sigma \ldot \psi = \inn(\sigma) \psi$.

The fundamental group $\pi_1$ of $E\smallsetminus S$ is described by the following generating set and relation:

\[\pi_1=\gen{\gamma_1,\dotsc,\gamma_{2g-2},\alpha_1,\alpha_2\scl \gamma_1\dotsm\gamma_{2g-2}=\alpha_1\alpha_2\alpha_1^{-1}\alpha_2^{-1}}.\]

For over $S$ simply branched covers, the image of each loop $\gamma_i$ under the monodromy map is a simple transposition $\tau_i$. Namely, there is over $b_i$ exactly one branch point of index $2$, and $\tau_i$ interchanges the two fiber points corresponding to the two sheets of the branching, leaving the other fiber points unchanged.

\begin{prop}
 The image of $\text{mon}$ is isomorphic as a $S_d$-set to \begin{align*}\widehat{T}_{g,d} = \{(\tau_1,\dotsc,\tau_{2g-2},\sigma_1,\sigma_2)\in S_d^{2g}\scl\\ \text{ each }\tau_i\text{ is a simple transposition, }\tau_1\dotsm\tau_{2g-2}=\sigma_1\sigma_2\sigma_1^{-1}\sigma_2^{-1}\}.\end{align*}
\end{prop}