\section{Introduction}
Let $E$ be an elliptical curve over $\Cbb$ and fix a genus $g\geq 1$.
The main interest of this thesis are the simply ramified morphisms $p\cl C\to E$, where $C$ is an arbitrary smooth irreducible complex curve of genus $g$. This thesis seeks to count such morphisms, in the way described below.

For any $d\geq 1$, the isomorphism classes of morphisms $p$ of degree $d$ may be classified by applying the theory of covering spaces in topology to the ramified covers (\ie the morphisms) of interest.
One finds that there are (up to isomorphism) finitely many such covers, so that one may ask about their number. Since a cover may have nontrivial automorphisms, a natural way to count covers would be to first apply the weighting $p\mapsto 1/\abs{\Aut(p)}$ to any cover $p$.
The goal is to study the generating series $F_g(q)$ over the weighted counts $N_{g,d}$ of (connected) covers of genus $g$ and degree $d$.

The main theorem of this thesis states that for $g\geq 2$, if $q$ is viewed as the complex variable $\exp(2\pi i\tau)$ and the generating series $F_g(q)$ as a function of $\tau$, then the function $F_g$ is a \emph{quasimodular form} of weight $6g-6$. This is one of the main theorems of \cite{Dijkgraaf}.

The concept of quasimodularity generalizes that of modularity; the definition here used comes from \cite{Kaneko-Zagier1995}. The same article also contains a theorem that will provide the last step to the proof of the main theorem. As for getting to the last step, \cite{Dijkgraaf} contains the sketch of an argument, expanded upon in \cite{Roth}, for turning the generating series $F_g$ into something more useful, \ie to which one may apply the theorem in \cite{Kaneko-Zagier1995}. The main part of this thesis follows the outline of \cite{Roth}.


\subsection*{Acknowledgements}
I would like to thank Prof.\ Daniel Huybrechts for his valuable advice during the writing of this thesis. I also warmly thank my family for their support throughout my studies.


\selectlanguage{ngerman}
\subsection*{Einleitung}
Sei $E$ eine elliptische Kurve über $\Cbb$, sei $g>1$ ein festes Geschlecht. Diese Arbeit befasst sich hauptsächlich mit den einfach verzweigten Morphismen $p\cl C\to E$, wobei $C$ eine beliebige glatte irreduzible komplexe Kurve vom Geschlecht $g$ ist. Ziel der Arbeit ist es, solche Morphismen auf die unten beschriebene Weise zu zählen.

Für $d>1$ können die Isomorphieklassen von Morphismen $p$ vom Grad $d$ klassifiziert werden, indem man topologische Überlagerungstheorie auf die zu untersuchende verzweigte Überlagerungen (also auf die Morphismen) anwendet. Es stellt sich heraus, dass es nur endlich viele solche Überlagerungen (bis auf Isomorphie) gibt, weshalb die Frage nach deren Anzahl aufkommt. Überlagerungen können nichttriviale Automorphismen haben; somit wäre eine natürliche Weise, die Überlagerungen $p$ zu zählen, dadurch gegeben, dass man zunächst die Gewichtung $p\mapsto 1/\abs{\Aut(p)}$ anwendet. Von Interesse ist die erzeugende Funktion $F_g(q)$ über die gewichteten Anzahlen $N_{g,d}$ der (zusammenhängenden) Überlagerungen vom Geschlecht $g$ und Grad $d$.

Das Hauptresultat dieser Arbeit besagt, dass für $g\geq 2$, falls man $q$ als die komplexe Variable $\exp(2\pi i\tau)$ und die erzeugende Reihe $F_g(q)$ als Funktion von $\tau$ auffasst, die Funktion $F_g$ eine \emph{quasimodulare Form} vom Gewicht $6g-6$ ist. Dieses Ergebnis ist eine der Hauptaussagen in \cite{Dijkgraaf}.

Der Begriff der Quasimodularität verallgemeinert den der Modularität, und ist aus \cite{Kaneko-Zagier1995} übernommen. Dieser Artikel liefert außerdem einen Satz, der den letzten Schritt des Beweises des Hauptresultats ermöglicht. Zum Erreichen dieses letztes Schrittes findet sich in \cite{Dijkgraaf} die Skizze eines Arguments, welches in \cite{Roth} ausgeführt wird. Mit diesem kann die erzeugende Reihe $F_g$ so umgeformt werden, dass sich der Satz aus \cite{Kaneko-Zagier1995} anwenden lässt. Der Hauptteil dieser Arbeit orientiert sich an \cite{Roth}.

\selectlanguage{english}