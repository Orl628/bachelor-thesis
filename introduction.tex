\section{Introduction}
Let $E$ be an elliptical curve over $\Cbb$ and fix a genus $g\geq 1$.
The main interest of this thesis are the morphisms $p\cl C\to E$, where $C$ is some (variable) smooth irreducible complex curve of genus $g$, that are simply ramified. This thesis seeks to count such morphisms, in the way described below.

For any $d\geq 1$, the equivalence classes of morphisms $p$ of degree $d$ may be classified by applying the theory of covering spaces in topology to the ramified covers (\ie the morphisms) of interest.
One finds that there are finitely many such covers, so that one may ask about their number. Since a cover may have automorphisms, a natural way to count covers would be to first apply the weighting $p\mapsto 1/\abs{\Aut(p)}$ to any cover $p$.
The goal becomes to study the generating series $F_g(q)$ over the weighted counts $N_{g,d}$ of (connected) covers of genus $g$ and degree $d$.

The main theorem of this thesis states that for $g\geq 2$, if $q$ is viewed as the complex variable $\exp(2\pi i\tau)$ and the generating series $F_g(q)$ as a function of $\tau$, then the function $F_g$ is a \emph{quasimodular form} of weight $6g-6$. This is one of the main theorems of \cite{Dijkgraaf}.

The concept of quasimodularity generalizes that of modularity; the definition here used is the one in \cite{Kaneko-Zagier1995}. This article also contains a theorem that will provide the last step to the proof of the main theorem. As for getting to the last step, \cite{Dijkgraaf} contains the sketch of an argument, expanded upon in \cite{Roth}, for turning the generating series $F_g$ into something more useful, \ie to which we may apply the theorem in \cite{Kaneko-Zagier1995}. The main part of this thesis follows the outline of \cite{Roth}.

\subsection*{Einleitung}

(Deutsche Übersetzung der endgültigen Einleitung)