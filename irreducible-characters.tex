\section{The group algebra of the symmetric group}

Let $\Cbb[\Sym_d]$ be the group algebra of the symmetric group, let $\mathcal{Z}_d$ be its centre. This is a commutative algebra, acting on itself linearly by multiplication. In this section, we relate this linear action to the matrix $M_d$ of the previous section, and we use the representation and character theory of the symmetric group to compute its eigenvalues.

\subsection{The centre of the group algebra}

\begin{defi}
 Let $\mc{Z}_d\subset\Cbb[\Sym_d]$ be the centre of the group algebra. If $c$ is a conjugacy class of $\Sym_d$, define the element $z_c\in\mc{Z}_d$ by
 \[
  z_c=\sum_{\sigma\in c}\sigma.
 \]
\end{defi}

\begin{rmk}
 The elements $z_c$ lie in the centre since $\alpha c=c\alpha$ for all conjugacy classes $c$ and elements $\alpha$ of $\Sym_d$. Further, the $z_c$ form a basis of $\mc{Z}_d$. Indeed, linear independence follows from the linear independence of the distinct elements $\sigma\in\Sym_d\subset\Cbb[\Sym_d]$. Furthermore, if $z\in\mc{Z}_d$, then the equalities $\alpha z\alpha^{-1}=z$ show that the $\Cbb$-coefficients of elements in the same conjugacy class are equal. Hence $\mc{Z}_d$ is $r$-dimensional, with $r=\partit{(d)}$.
\end{rmk}

Recall the definition of $M_d$ from the previous section. There, we fixed a system of representatives for the conjugacy classes of $\Sym_d$. However, since the definition does  not depend on the chosen representatives, we may also define $M_d$ to be a matrix indexed by the conjugacy classes of $\Sym_d$, ordered in the same way as before. The new, equivalent definition is as follows.

\begin{defi} Let $c\pr,c$ be conjugacy classes of $\Sym_d$. Define the matrix $M_d$ by
\[(M_d)_{c\pr\nscms c} = |\{\tau\scl \tau\text{ is a transposition such that }\tau\sigma\in c\pr\}|,\] where $\sigma$ is any representative of $c$.
\end{defi}

From now on, we choose the ordering of the basis $\{z_c\}_c$ and the ordering of the columns of $M_d$ to be compatible, \ie coming from the same fixed ordering of the conjugacy classes $\{c\}$.

\begin{prop} \label{prop:transpose-correspondence}
 Let $t$ be the conjugacy class containing all transpositions. Let $z_t$ be the corresponding basis element of $\mc{Z}_d$. Let $M_t$ be the size $r$ square matrix representing the $\Cbb$-linear map $(z_t\,\cdot):\mc{Z}_d\to\mc{Z}_d$ given by multiplication with $z_t$. Then the matrix $M_t$ is the transpose of $M_d$.
\end{prop}
\begin{proof}
 Let $c,c\pr$ be conjugacy classes. Note that if $z=\sum_{\sigma\in\Sym_d}\lam_\sigma \sigma=\sum_{\tilde c}\lam_{\tilde c} z_{\tilde c}$, then the coefficient $\lam_{\tilde c}$ is equal to the coefficient $\lam_{\sigma}$, for any $\sigma\in \tilde c$. Now let $\sigma\in c$, and consider the product
 \[
  z_tz_{c\pr}=\bigl(\sum_{\tau\in t}\tau\bigr)\bigl(\sum_{\sigma\pr\in c\pr}\sigma\pr\bigr)=\sum_{\sigma\in\Sym_d}\;\sum_{\tau\sigma\in c'}\sigma.
 \]
 In this expansion, the coefficient $\lam_{\sigma}$ of any element $\sigma\in c$ is the quantity \[|\{\tau\in t\scl \tau\sigma\in c\pr\}|.\] It follows that $(M_t)_{c\pr\nscms c}=\lambda_c=\lambda_{\sigma}=(M_d)_{c\cms c\pr}$.
\end{proof}

\subsection{Irreducible characters of the symmetric group}

We have reduced our problem of computing the eigenvalues of $M_d$ to the computation of the eigenvalues of $M_t$. More generally, we find that $\mc{Z}_d$ actually has a basis $\{w_\chi\}$, indexed by the irreducible characters of $\Sym_d$, such that each $w_\chi$ is an eigenvector for all linear maps defined by multiplication with any element of $\mc{Z}_d$, and such that the corresponding eigenvalues are easy to compute.

\begin{defi} \ 

  1. Let $\rho$ be an irreducible representation of $\Cbb[\Sym_d]$, \ie a group homomorphism $\rho\cl \Sym_d\to \GL(\Cbb^n)$ such that for each $\sigma\in\Sym_d$ there are no $\rho(\sigma)$-invariant subspaces. The \emph{irreducible character associated to $\rho$} is defined as the map $\chi_\rho\cl \Sym_d\to\Cbb\cms \sigma\mapsto\Tr(\rho(\sigma)).$
  
  2. An \emph{irreducible character} of $\Sym_d$ is a map $\chi\cl \Sym_d\to\Cbb$ of the form $\chi=\chi_\rho$ for some irreducible representation $\rho$. Its \emph{dimension} $\dim(\chi)$ is defined as the dimension of the associated representation $\dim\rho=\chi(1)$.
\end{defi}

For brevity, we will refer to irreducible characters simply as characters. 

\begin{rmk}
 Characters are constant on conjugacy classes. It is therefore justified to write $\chi(c)\in\Cbb$ for a character $\chi$ and a conjugacy class $c$.
\end{rmk}

\begin{rmk}
 The number of irreducible representations of a finite group, up to isomorphism, is equal to the number of its conjugacy classes (see for example \cite[p. 19, Thm. 7]{Serre77}). In the case of the symmetric group, both the set of conjugacy classes and the set of irreducible representations are indexed by the set of Young diagrams, in a natural way. The irreducible representations are recovered from the Young diagrams via Specht modules.
\end{rmk}

\begin{prop}\label{pr:orthogonalities}
 Let $\chi, \chi\pr$ be characters, let $\sigma_1\in\Sym_d$. Then
 \[
  \sum_{\sigma\in\Sym_d}\chi(\sigma)\chi\pr(\sigma^{-1}\sigma_1) = 
   \begin{cases} \frac{d!}{\dim(\chi)}\chi(\sigma_1) &\mbox{if } \chi = \chi\pr \\
                                     0 &\mbox{else.}
   \end{cases} 
 \]
 Further for conjugacy classes $c$ and $c_1$ we have
 \[
  \sum_{\chi}\chi(c)\chi(c_1^{-1}) =
   \begin{cases} \frac{d!}{\abs{c}} &\mbox{if } c = c\pr \\
                                  0 &\mbox{else,}
   \end{cases}
 \]
 where the $\chi$ runs through the irreducible characters of $\Sym_d$.
\end{prop}
\begin{proof}
 See \cite[Thm.\ 2.13 and Thm.\ 2.18]{Isaacs}.
\end{proof}

\begin{defi}
 Let $\chi$ be a character of $\Sym_d$. Define the element $w_\chi \in \mc{Z}_d$ by
 \[
  w_\chi = \frac{\dim(\chi)}{d!}\sum_{c}\chi(c^{-1})z_c = \frac{\dim(\chi)}{d!}\sum_{\sigma\in\Sym_d}\chi(\sigma^{-1})\sigma.
 \]
\end{defi}

\begin{prop} \label{prop:simultaneous-eigenvectors}
 The $w_\chi$ form a basis of $\mc{Z}_d$. With respect to this basis, if $z=\sum_\chi a_\chi w_\chi$ is any element of $\mc{Z}_d$, then the linear map $(z\,\cdot)$ is represented by the matrix $\Diag((a_\chi)_\chi)$. Moreover, the matrix representing the map $(z_t\,\cdot)$ has the diagonal entries $a_\chi=\binom{d}{2} \chi({t})/\dim({\chi})$. 
\end{prop}
\begin{proof}
 The two formulae in Proposition \ref{pr:orthogonalities} lead to the formulae
 \begin{equation} \label{eq:lin-indep}
  w_\chi w_{\chi\pr} =
  \begin{cases} w_\chi &\mbox{if } \chi = \chi\pr \\
                     0 &\mbox{else}
  \end{cases}
 \end{equation}
 and
 \begin{equation} \label{eq:gen-sys}
  z_c=\sum_{\chi}\left(\frac{\abs{c}\chi(c)}{\dim(\chi)}\right)w_\chi
 \end{equation}
 respectively.
 By \eqref{eq:lin-indep}, the $w_\chi$ are linearly independent (multiply a linear relation with one of the $w_\chi$), and by \eqref{eq:gen-sys} they span $\mc{Z}_d$.
 The second statement follows directly from \eqref{eq:lin-indep}. The last statement follows with \eqref{eq:gen-sys} from $t=t^{-1}$ and $|t|=\binom{d}{2}$.
\end{proof}

\begin{lemma} \label{prop:eigenvalues}
 With the notation from Lemma \ref{prop:counting-fct}, the eigenvalues of $M_d$ are given by \[\mu_{i,d}=\frac{\binom{d}{2} \chi(t)}{\dim(\chi)},\] where $\chi$ is the $i$-th character and $t$ is the conjugation class of $\Sym_d$ containing all transpositions.
\end{lemma}
\begin{proof}
 By Proposition \ref{prop:transpose-correspondence}, the eigenvalues of $M_d$ are the same as the eigenvalues of $M_t$. By Proposition \ref{prop:simultaneous-eigenvectors}, the $w_\chi$ are eigenvectors for the map $(z_t\,\cdot)$, represenented by the matrix $M_t$. The eigenvalues are given by the $a_\chi$ from the same proposition.
\end{proof}