\section{Modular curves}

\newcommand{\jrm}{\mathrm{j}}

One may weaken the definition of a modular form by requiring that modular condition be met only for transformations lying in certain subgroups $\Gamma$ of $\SL_2(\Zbb)$. In this section we will calculate the dimension of the associated space of weight $k$ modular forms $\M_k(\Gamma)$ using the fact that modular forms can be seen as section of a certain line bundle on a special Riemann surface, the modular curve associated to the subgroup $\Gamma$. We will roughly follow \cite[Ch. 1-3]{diamond2006}.

\subsection{Congruence subgroups and modular forms}

\begin{defi}
 Let $N\in \Zbb$.
 
 1. The \emph{principal congruence subgroup} of \emph{level} $N$ is the subgroup
 \[\Gamma(N)=\left\{\begin{pmatrix} a & b \\ c & d \end{pmatrix}\in\SL_2(\Zbb)\scl \begin{pmatrix}a & b \\ c & d\end{pmatrix} \equiv \begin{pmatrix}1 & 0 \\ 0 & 1\end{pmatrix} \pmod{N}\right\}.\]
 
 2. A \emph{congruence subgroup} is a subgroup $\Gamma\subset\SL_2(\Zbb)$ such that $\Gamma(N)\subset\Gamma$ for some $N\in\Zbb$. We then say that $\Gamma$ has \emph{level} $N$. 
\end{defi}

\begin{rmk}
 The subgroup $\Gamma(N)$ is the kernel of the component-wise congruence map $\SL_2(\Zbb)\to\SL_2(\Zbb/N\Zbb)$. It is hence normal in $\SL_2(\Zbb)$ and of finite index. Consequently, each congruence subgroup has finite index in $\SL_2(\Zbb)$, while not being necessarily normal.
\end{rmk}

\begin{defi}
 Let $\gamma = \left(\begin{smallmatrix}a & b \\ c & d\end{smallmatrix}\right)$ be an element of $\GL_2(\Cbb)$ and let $f\cl\mc{H}\to\what{\Cbb}$ be a holomorphic function. For $\tau\in\mc{H}$ define the \emph{factor of automorphy}
 \[\mathrm{j}(\gamma,\tau)\defeq c\tau+d\]
 and for $k\in \Zbb$ the function $f[\gamma]_k\cl\mc{H}\to\Cbb$ by
 \[f[\gamma]_k(\tau) \defeq \det(\gamma)^{k/2}\jrm(\gamma,\tau)^{-k}f(\gamma\tau).\]
\end{defi}

\begin{remark}
 Let $\gamma,\gamma'\in\SL_2(\Zbb)$ and $\tau\in\mc{H}$.
 
 1. The factor of automorphy satisfies $\jrm(\gamma\gamma',\tau) = \jrm(\gamma,\gamma'(\tau))\jrm(\gamma',\tau)$.
 
 2. For all holomorphic functions $f:\mc{H}\to\Cbb$, we have $f[\gamma\gamma']_k=(f[\gamma]_k)[\gamma']_k$.
\end{remark}

\begin{defi} \label{df:modular-forms} Let $f\cl \mc{H} \to \Cbb$ be a function, let $\Gamma$ be a congruence subgroup. Hence $\Gamma$ contains an element of the form $\left(\begin{smallmatrix}1 & *\\0 & 1\end{smallmatrix}\right)\in\Gamma$. Let $h\in\Nbb$ be minimal with the property that $\left(\begin{smallmatrix}1 & h\\0 & 1\end{smallmatrix}\right)\in\Gamma$. For $\tau\in\mc{H}$, set $q_h=\exp(2\pi i\tau/h)$.

  1. The function $f$ is \emph{$h\Zbb$-periodic}, if it satisfies $f(\tau + h) = f(\tau)$ for all $\tau \in \mc{H}$. Analogously to the case $h=1$, there exists a function $\tilde f\cl B\setminus \{0\} \to \Cbb$ such that $f(\tau)=\tilde f(q_h)$ for all $\tau$. Now let $f$ be holomorphic, so that $\tilde f$ is also holomorphic. We say that $f$ is \emph{holomorphic at infinity}, if $\tilde f$ has a holomorphic continuation to the whole of $B$.
  
  2. The function $f$ is said to satisfy the \emph{modular condition of weight $k$ with respect to $\Gamma$}, if $f[\gamma]_k=f$ for all $\gamma \in \Gamma$. Such a function is $h\Zbb$-periodic, since $\left(\begin{smallmatrix}1 & h\\0 & 1\end{smallmatrix}\right)\in\Gamma$.
  
  3. For all $\alpha\in\SL_2(\Zbb)$, the group $\alpha^{-1}\Gamma\alpha$ is a conjugation subgroup. Now assume that the function $f$ is holomorphic and that it satisfies the modular condition of weight $k$ with respect to $\Gamma$. Hence, for all $\alpha$, the function $f[\alpha]_k$ satisfies the modular condition of the same weight with respect to $\alpha^{-1}\Gamma\alpha$, and is hence $h_\alpha\Zbb$-periodic for some $h_\alpha$. We define $f$ to be \emph{holomorphic at all cusps of $\Gamma$}, if for all $\alpha\in\SL_2(\Zbb)$, the function $f[\alpha]_k$ is holomorphic at $\infty$.
  
  4. The function $f$ is a \emph{modular form (of weight $k$) with respect to $\Gamma$} if it is holomorphic, satisfies the modular condition and is holomorphic at all cusps of $\Gamma$.
  
  5. The function $f$ is a \emph{cusp form of weight $k$ with respect to $\Gamma$}, if it is a modular form of weight $k$, and if the associated holomorphic function $\tilde{f}$ satisfies $\tilde{f}(0)=0$ after its holomorphic continuation. 
\end{defi}

\subsection{The topology of modular curves}

Let $\Gamma$ be a congruence subgroup.

\begin{defi} \ 

  1. For $M>0$, define the set $\mc{N}_M \subset \Cbb\cup\{\infty\}$ by
  \[\mc{N}_M=\left\{\tau\in\mc{H}\scl \Img(\tau)>M\right\}\cup\{\infty\}.\]

  2. Define the \emph{compact upper half-plane} $\mc{H}^*$ to be the set
  \[\mc{H}^*=\mc{H}\cup\Qbb\cup\{\infty\},\]
  endowed with the topology generated by union of the topology of $\mc{H}$ with the set
  \[\left\{\alpha(\mc{N_M})\scl \alpha\in\SL_2(\Qbb), M\in\Rbb_{>0}\right\}.\]

  With this definition, the group $\SL_2(\Zbb)$ acts on $\mc{H}^*$ by continuous maps. This group action is transitive on the subset $\Qbb\cup\{\infty\}$.

  3. The \emph{modular curve} $X(\Gamma)$ is defined as the quotient space $_\Gamma\backslash\mc{H}^*$. Denote he canonical projection map $\mc{H}^*\to X(\Gamma)$ by $\pi$.
\end{defi}

For $\tau\in\mc{H}^*$, denote the stabilizer of $\tau$ under the action of $\Gamma$ by $\Gamma_\tau$.

\begin{rmk}\label{pr:real-op} \ 

 \itemrm{1.} Define the function $s\cl\mc{H}\to\SL_2(\Rbb)$ by $s(x+iy)=\frac{1}{\sqrt{y}}\left(\begin{smallmatrix}y & x \\ 0 & 1\end{smallmatrix}\right)$. If $\tau\in\mc{H}$, then $s(\tau)(i)=\tau$.
 
 \itemrm{2.} The stabilizer of $i$ with respect to the transitive group action of the topological group $\SL_2(\Rbb)$ on $\mc{H}$ is the compact subgroup $\SO_2(\Rbb)$.
 
 \itemrm{3.} Let $e_1,e_2\in\mc{H}$. We have $\gamma(e_1)=e_2$ if and only if $\gamma\in s(e_2)\SO_2(\Rbb)s(e_1)^{-1}.$
\end{rmk}

\begin{prop}
 If $\tau_1, \tau_2\in\mc{H}$, then there are open neighborhoods $U'_1$ of $\tau_1$ and $U'_2$ of $\tau_2$, such that for all but finitely many $\gamma\in\SL_2(\Zbb)$, the sets $\gamma(U'_1)$ and $U'_2$ do not meet.
\end{prop}

\begin{proof}
 Choose $U'_1$ and $U'_2$ to be any open neighborhoods belonging to the topology of $\mc{H}$, and with compact closure. Let $\gamma\in\SL_2(\Zbb)$. The previous remark implies that $\gamma(\closure{U_1'})\cap\closure{U_2'}\neq\emptyset$ is equivalent to \[\gamma \in \SL_2(\Zbb) \cap \bigcap_{\substack{e_1\in\closure{U_1'}\\e_2\in\closure{U_2'}}}s(e_2)\SO_2(\Rbb)s(e_1)^{-1}.\] But this subgroup is compact and discrete, hence finite.
\end{proof}

\begin{prop}
 Let $\tau_1$ and $\tau_2$ be elements of $\mc{H}^*.$ Then there exist open neighborhoods $U_1$ of $\tau_1$ and $U_2$ of $\tau_2$ such that for all $\gamma\in\SL_2(\Zbb)$, if $\gamma(U_1)$ meets $U_2$ then $\gamma\tau_1 = \tau_2$.
\end{prop}

\begin{proof}
 We consider three cases separately.
 
 \itemrm{1.} Let $\tau_1,\tau_2\in\mc{H}$. Let $U'_1$ and $U'_2$ be open neighborhoods of $\tau_1$ and $\tau_2$, respectively, such that the set
 \[\{\gamma\in\SL_2(\Zbb)\scl \gamma(U'_1)\cap U'_2\neq\emptyset, \gamma(\tau_1)\neq \tau_2\}\]
 is finite. We denote this set by $F$. For each $\gamma\in F$, choose disjoint open neighborhoods $U_{1,\gamma}$ and $U_{2,\gamma}$ of $\gamma\tau_1$ and $\tau_2$, respectively, and put
 \[U_1 = U'_1\cap\left(\bigcap_{\gamma\in F}\gamma^{-1}(U_{1,\gamma})\right)\text{ and}\]
 \[U_2 = U'_2\cap\left(\bigcap_{\gamma\in F}U_{2,\gamma}\right).\]
 Then $U_1$ and $U_2$ satisfy the required propertieties.
 
 \itemrm{2.} Let $\tau_1\in\Qbb\cup\{\infty\}$ and $\tau_2\in\mc{H}$. Choose $U_2$ to be any open neighborhood in $\mc{H}$ with compact closure. Now, there is some $M\geq 0$ such that $\SL_2(\Zbb)\bar{U}_2\cup\mc{N}_M=\emptyset$. Let $\alpha\in\SL_2(\Zbb)$ such that $\alpha(\infty)=\tau_1$. Choose $U_1=\alpha(\mc{N}_M)$. Then for all $\gamma\in\SL_2(\Zbb)$, the sets $\gamma(U_2)$ and $U_1$ are disjoint.
 
 \itemrm{3.} Let $\tau_1,\tau_2\in\Qbb\cup\{\infty\}$. Let $\alpha_1,\alpha_2\in\SL_2(\Zbb)$ such that $\alpha_1(\infty)=\tau_1$ and $\alpha_2(\infty)=\tau_2$. Choose $U_1=\alpha_1(\mc{N}_2)$ and $U_2=\alpha_2(\mc{N}_2)$. Then $U_1$ and $U_2$ satisfy the required properties.
\end{proof}

\begin{cor} \label{pr:chart-nhood}
 Let $\tau\in\mc{H}^*.$ Then there is an open neighborhood $U$ of $\tau$ such that for all $\gamma\in\mc{H}$, if $\gamma(U)$ meets $U$ then $\gamma\in\Gamma_\tau$.
\end{cor}

\begin{prop}
 The modular curve $X(\Gamma)$ connected, compact, and Hausdorff.
\end{prop}

\begin{proof}
 The connectedness of $X(\Gamma)$ follows from the connectedness of $\mc{H}^*$. For compactness, define the subsets
 \[\mc{D}=\{\tau\in\mc{H}\scl \abs{\tau}\geq 1, \Rea(\tau)\leq 1/2\}\] and $\mc{D}^*=\mc{D}\cup\{\infty\}$. The subset $\mc{D}^*\subset\mc{H}^*$ is compact and a fundamental domain for the $\SL_2(\Zbb)$-action on $\mc{H}^*$. Since $\Gamma$ has finite index in $\SL_2(\Zbb)$, it follows that a possible fundamental domain for the $\Gamma$-action is given by the union of finitely many images of $\mc{D}^*$ under elements of $\SL_2(\Zbb)$. Therefore, the modular curve $X(\Gamma)$ is compact. Finally, the Hausdorff property follows from the previous proposition. 
\end{proof}

\subsection{Modular curves as Riemann surfaces}

The modular curve $X(\Gamma)$ may be given the structure of a Riemann surface. The needed local data is summarized below. We use the following convention: a subgroup $G$ of $\SL_2(\Zbb)$ need not contain the matrix $-\id$; we denote the subroup generated by $G$ and $\{-\id\}$ by $\pm G$.

\begin{prop} \ 

 \itemrm{1.} For $\tau\in\mc{H}$, the isotropy group $\Gamma_\tau$ is finite cyclic.
 
 \itemrm{2.} For $s\in\Qbb\cup\{\infty\}$, the isotropy group $\Gamma_s$ has finite index in the isotropy group $\SL_2(\Zbb)_s$.
\end{prop}

\begin{defi} \

 \itemrm{1.} Let $\tau\in\mc{H}$. The \emph{period} of $\tau$ is the number \[h_\tau=\abs{\!\pm\Gamma_\tau/\{\pm\id\}}\] of maps in the isotropy group of $\tau$. The period of $\tau$ only depends on the class $\Gamma\tau$. If $h_\tau > 1$, then we call the point $\tau$, or interchangeably the point $\pi(\tau)$, an \emph{elliptic point} for $\Gamma$. We further define the map $\delta_\tau\cl\mc{H}^*\to\Cbb$ to be the map represented by the matrix
 \[\delta_\tau = \begin{pmatrix}1 & -\tau \\ 1 & -\conj{\tau}\end{pmatrix}\in\GL_2(\Cbb)\] and the map $\rho_\tau\cl\Cbb\to\Cbb$ by $\rho_\tau(z)=z^{h_\tau}.$
 
 \itemrm{2.} Let $s\in\Qbb\cup\{\infty\}$. The \emph{width} of $s$ is the number \[h_s=[\SL_2(\Zbb)_s:\pm\Gamma_\tau].\] This also only depends on the class $\Gamma s$. We call the point $s$, or interchangeably the point $\pi(s)$, a \emph{cusp} for $\Gamma$. Furthermore, we define $\delta_s\in\SL_2(\Zbb)$ to be any map taking $\infty$ to $s$. Finally, define the map $\rho_s(z)\cl\mc{H}^*\to\Cbb$ by $\rho_s(z)=\exp(2\pi iz/h_s)$. Note that this is well-defined on $\infty$ since we restrict to the upper half-plane.
\end{defi}

\begin{rmk}
 There are only two elliptic points on $X(\SL_2(\Zbb))$, namely the $\SL_2(\Zbb)$-classes of $i$ and $\mu_3$, where $\mu_3=\exp(2\pi i/3)$. Their periods are $2$ and $3$, respectively. Since $\Gamma_\tau\subseteq\SL_2(\Zbb)_\tau$, each elliptic point $\tau$ for $\Gamma$ has period $2$ or $3$, and lies in one of the classes $\Gamma i$ or $\Gamma \mu_3$, according to wether its period is $2$ or $3$. Since $\Gamma$ has finite index in $\SL_2(\Zbb)$, it follows that there are only finitely many elliptic points on $X(\Gamma)$.
 
 The only cusp of $\SL_2(\Zbb)$ is $\infty$, whose isotropy subgroup is the group generated by $\smatrix{1&1\\0&1}$. Hence the width of $\infty$ with respect to $\Gamma$ is the smallest $h$ such that $\smatrix{1&h\\0&1}\in\Gamma$, \cf Definition \ref{df:modular-forms}.
\end{rmk}

For $\tau\in\mc{H}^*$, let $U_\tau$ be an open neighborhood of $\tau$ such that for all $\gamma\in\mc{H}$, if $\gamma(U_\tau)$ meets $U_\tau$ then $\gamma\in\Gamma_\tau$, as per Corollary \ref{pr:chart-nhood}. If $\tau$ is an elliptic point or a cusp for $\Gamma$, we may assume that $U_\tau$ contains no further preimages of elliptic points or cusps. If $\pi(\tau)$ is neither, we may assume that $U_\tau$ contains no ellliptic points or cusps altogether.

Define the map $\varphi_\tau=\rho_\tau\circ\delta_\tau|_{U_\tau}$. This is a map with open image in $\Cbb$. For $\tau_1,\tau_2\in U_\tau$, we have $\varphi_\tau(\tau_1)=\varphi_\tau(\tau_2)$ if and only if $\pi(\tau_1)=\pi(\tau_2)$. Hence, the map $\varphi_\tau$ induces an injective continuous map $\psi_\tau\cl U_\tau\to\Cbb$. We take $(\pi(U_\tau),\phi_\tau)$ to be our chosen chart around $\pi(\tau)$.

\begin{prop}
 The charts $(\pi(U_\tau),\psi_\tau)_{\tau\in\mc{H}^*}$ form a holomorphic atlas for the modular curve $X(\Gamma)$.
\end{prop}

\begin{prop}
 Let $\Gamma_1$ and $\Gamma_2$ be two conjugation subgroups with $\Gamma_1\subseteq\Gamma_2$. If $-\id\in\Gamma_2\ssm\Gamma_1$, then the induced morphism $X(\Gamma_1)\to X(\Gamma_2)$ has degree $[\Gamma_2:\Gamma_1]/2$, else it has degree $[\Gamma_2:\Gamma_1]$. The ramification index of a point $\pi_1(\tau)\in X(\Gamma_1)$ is $[\pm\Gamma_{2,\tau}:\pm\Gamma_{1,\tau}]$.
\end{prop}

\begin{cor}
 Let $f\cl X(\Gamma)\to X(\SL_2(\Zbb))$ be the morphism induced by the inclusion $\Gamma\in\SL_2(\Zbb)$. The ramification index of a point $\pi(\tau)\in X(\Gamma)$ is $h_\tau$, if $\tau$ is a cusp for $\Gamma$ or if $\tau$ is an elliptic point for $\SL_2(\Zbb)$ but not for $\Gamma$. Else, the ramification index is $1$.
\end{cor}

Let $y_2$, $y_3$, and $y_\infty$ be the images of $i$, $\mu_3$, and $\infty$, respectively, under the projection $\mc{H}^*\to\ X(\SL_2(\Zbb))$. For $h\in\{2,3\}$, let $\varepsilon_h$ denote the number of elliptic points in $X(\Gamma)$ of period $h$. Further, denote by $\varepsilon_\infty$ the number of cusps in $X(\Gamma)$.

The ramification locus of $f$ is contained in the set ${y_1,y_2,y_\infty}$. Now let $d$ be the degree of $f$. By applying the previous corollary and the formula 
\[d=\sum_{x\in f^{-1}(y_h)}e_f(x),\] one verifies the formulae
\[\sum_{x\in f^{-1}(y_h)}(e_f(x)-1)=\frac{h-1}{h}(d-\varepsilon_h),\]
\[\sum_{x\in f^{-1}(\infty)}(e_f(x)-1)=d-\varepsilon_\infty.\]

\begin{prop}
 The genus $g$ of the modular curve $X(\Gamma)$ is given by
 \[g=1+\frac{d}{12}-\frac{\varepsilon_2}{4}-\frac{\varepsilon_3}{3}-\frac{\varepsilon_\infty}{2}.\]
\end{prop}

\begin{proof}
 The statement follows from the Riemann--Hurwitz formula, the previous discussion, and the fact that $X(\SL_2(\Zbb))$ has genus $0$, which will be proved later in Example \ref{ex:j-invariant}.
\end{proof}






