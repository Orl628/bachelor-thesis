\section{Modular curves}

\newcommand{\jrm}{\mathrm{j}}

One may weaken the definition of a modular form by requiring that modular condition be met only for transformations lying in certain subgroups $\Gamma$ of $\SL_2(\Zbb)$. In this section we will calculate the dimension of the associated space of weight $k$ modular forms $\Mrm_k(\Gamma)$ using the fact that modular forms can be seen as section of a certain line bundle on a special Riemann surface, the modular curve associated to the subgroup $\Gamma$. We will roughly follow \cite[Ch. 1-3]{diamond2006}.

\subsection{Congruence subgroups and modular curves}

\begin{defi}
 Let $N\in \Zbb$.
 
 1. The \emph{principal congruence subgroup} of \emph{level} $N$ is the subgroup
 \[\Gamma(N)=\left\{\begin{pmatrix} a & b \\ c & d \end{pmatrix}\in\SL_2(\Zbb)\scl \begin{pmatrix}a & b \\ c & d\end{pmatrix} \equiv \begin{pmatrix}1 & 0 \\ 0 & 1\end{pmatrix} \pmod{N}\right\}.\]
 
 2. A \emph{congruence subgroup} is a subgroup $\Gamma\subset\SL_2(\Zbb)$ such that $\Gamma(N)\subset\Gamma$ for some $N\in\Zbb$. We then say that $\Gamma$ has \emph{level} $N$. 
\end{defi}

\begin{rmk}
 The subgroup $\Gamma(N)$ is the kernel of the component-wise congruence map $\SL_2(\Zbb)\to\SL_2(\Zbb/N\Zbb)$. It is hence normal in $\SL_2(\Zbb)$ and of finite index. Consequently, each congruence subgroup has finite index in $\SL_2(\Zbb)$, while not being necessarily normal.
\end{rmk}

\begin{rmk} \label{pr:smallest-h}
 Since $\Gamma(N)\subset\Gamma$ for some $N$, each congruence subgroup $\Gamma$ contains an element of the form $\left(\begin{smallmatrix}1 & *\\0 & 1\end{smallmatrix}\right)$.
\end{rmk}

For the rest of the section, fix a congruence subgroup $\Gamma$.

\begin{defi} \ 

  1. For $M>0$, define the set $\mc{N}_M \subset \Cbb\cup\{\infty\}$ by
  \[\mc{N}_M=\left\{\tau\in\mc{H}\scl \Img(\tau)>M\right\}\cup\{\infty\}.\]

  2. Define the \emph{compact upper half-plane} $\mc{H}^*$ to be the set
  \[\mc{H}^*=\mc{H}\cup\Qbb\cup\{\infty\},\]
  endowed with the topology generated by union of the topology of $\mc{H}$ with the set
  \[\left\{\alpha(\mc{N_M})\scl \alpha\in\SL_2(\Qbb), M\in\Rbb_{>0}\right\}.\]

  With this definition, the group $\SL_2(\Zbb)$ acts on $\mc{H}^*$ by continuous maps. This group action is transitive on the subset $\Qbb\cup\{\infty\}$.

  3. The \emph{modular curve} $X(\Gamma)$ is defined as the quotient space $_\Gamma\backslash\mc{H}^*$. Denote he canonical projection map $\mc{H}^*\to X(\Gamma)$ by $\pi$.
\end{defi}

For $\tau\in\mc{H}^*$, denote the stabilizer of $\tau$ under the action of $\Gamma$ by $\Gamma_\tau$.

\begin{rmk}\label{pr:real-op} \ 

 \itemrm{1.} Define the function $s\cl\mc{H}\to\SL_2(\Rbb)$ by $s(x+iy)=\frac{1}{\sqrt{y}}\left(\begin{smallmatrix}y & x \\ 0 & 1\end{smallmatrix}\right)$. If $\tau\in\mc{H}$, then $s(\tau)(i)=\tau$.
 
 \itemrm{2.} The stabilizer of $i$ with respect to the transitive group action of the topological group $\SL_2(\Rbb)$ on $\mc{H}$ is the compact subgroup $\SO_2(\Rbb)$.
 
 \itemrm{3.} Let $e_1,e_2\in\mc{H}$. We have $\gamma(e_1)=e_2$ if and only if $\gamma\in s(e_2)\SO_2(\Rbb)s(e_1)^{-1}.$
\end{rmk}

\begin{prop}
 If $\tau_1, \tau_2\in\mc{H}$, then there are open neighborhoods $U'_1$ of $\tau_1$ and $U'_2$ of $\tau_2$, such that for all but finitely many $\gamma\in\SL_2(\Zbb)$, the sets $\gamma(U'_1)$ and $U'_2$ do not meet.
\end{prop}

\begin{proof}
 Choose $U'_1$ and $U'_2$ to be any open neighborhoods belonging to the topology of $\mc{H}$, and with compact closure. Let $\gamma\in\SL_2(\Zbb)$. The previous remark implies that $\gamma(\closure{U_1'})\cap\closure{U_2'}\neq\emptyset$ is equivalent to \[\gamma \in \SL_2(\Zbb) \cap \bigcap_{\substack{e_1\in\closure{U_1'}\\e_2\in\closure{U_2'}}}s(e_2)\SO_2(\Rbb)s(e_1)^{-1}.\] But this subgroup is compact and discrete, hence finite.
\end{proof}

\begin{prop}
 Let $\tau_1$ and $\tau_2$ be elements of $\mc{H}^*.$ Then there exist open neighborhoods $U_1$ of $\tau_1$ and $U_2$ of $\tau_2$ such that for all $\gamma\in\SL_2(\Zbb)$, if $\gamma(U_1)$ meets $U_2$ then $\gamma\tau_1 = \tau_2$.
\end{prop}

\begin{proof}
 We consider three cases separately.
 
 \itemrm{1.} Let $\tau_1,\tau_2\in\mc{H}$. Let $U'_1$ and $U'_2$ be open neighborhoods of $\tau_1$ and $\tau_2$, respectively, such that the set
 \[\{\gamma\in\SL_2(\Zbb)\scl \gamma(U'_1)\cap U'_2\neq\emptyset, \gamma(\tau_1)\neq \tau_2\}\]
 is finite. We denote this set by $F$. For each $\gamma\in F$, choose disjoint open neighborhoods $U_{1,\gamma}$ and $U_{2,\gamma}$ of $\gamma\tau_1$ and $\tau_2$, respectively, and put
 \[U_1 = U'_1\cap\left(\bigcap_{\gamma\in F}\gamma^{-1}(U_{1,\gamma})\right)\text{ and}\]
 \[U_2 = U'_2\cap\left(\bigcap_{\gamma\in F}U_{2,\gamma}\right).\]
 Then $U_1$ and $U_2$ satisfy the required propertieties.
 
 \itemrm{2.} Let $\tau_1\in\Qbb\cup\{\infty\}$ and $\tau_2\in\mc{H}$. Choose $U_2$ to be any open neighborhood in $\mc{H}$ with compact closure. Now, there is some $M\geq 0$ such that $\SL_2(\Zbb)\bar{U}_2\cup\mc{N}_M=\emptyset$. Let $\alpha\in\SL_2(\Zbb)$ such that $\alpha(\infty)=\tau_1$. Choose $U_1=\alpha(\mc{N}_M)$. Then for all $\gamma\in\SL_2(\Zbb)$, the sets $\gamma(U_2)$ and $U_1$ are disjoint.
 
 \itemrm{3.} Let $\tau_1,\tau_2\in\Qbb\cup\{\infty\}$. Let $\alpha_1,\alpha_2\in\SL_2(\Zbb)$ such that $\alpha_1(\infty)=\tau_1$ and $\alpha_2(\infty)=\tau_2$. Choose $U_1=\alpha_1(\mc{N}_2)$ and $U_2=\alpha_2(\mc{N}_2)$. Then $U_1$ and $U_2$ satisfy the required properties.
\end{proof}

\begin{cor} \label{pr:chart-nhood}
 Let $\tau\in\mc{H}^*.$ Then there is an open neighborhood $U$ of $\tau$ such that for all $\gamma\in\mc{H}$, if $\gamma(U)$ meets $U$ then $\gamma\in\Gamma_\tau$.
\end{cor}

\begin{prop}
 The modular curve $X(\Gamma)$ connected, compact, and Hausdorff.
\end{prop}

\begin{proof}
 The connectedness of $X(\Gamma)$ follows from the connectedness of $\mc{H}^*$. For compactness, define the subsets
 \[\mc{D}=\{\tau\in\mc{H}\scl \abs{\tau}\geq 1, \Rea(\tau)\leq 1/2\}\] and $\mc{D}^*=\mc{D}\cup\{\infty\}$. The subset $\mc{D}^*\subset\mc{H}^*$ is compact and a fundamental domain for the $\SL_2(\Zbb)$-action on $\mc{H}^*$. Since $\Gamma$ has finite index in $\SL_2(\Zbb)$, it follows that a possible fundamental domain for the $\Gamma$-action is given by the union of finitely many images of $\mc{D}^*$ under elements of $\SL_2(\Zbb)$. Therefore, the modular curve $X(\Gamma)$ is compact. Finally, the Hausdorff property follows from the previous proposition. 
\end{proof}

\subsection{Modular curves as Riemann surfaces}

The modular curve $X(\Gamma)$ may be given the structure of a Riemann surface. The needed local data is summarized below. We use the following convention: a subgroup $G$ of $\SL_2(\Zbb)$ need not contain the matrix $-\id$; we denote the subroup generated by $G$ and $\{-\id\}$ by $\pm G$.

\begin{prop} \ 

 \itemrm{1.} For $\tau\in\mc{H}$, the isotropy group $\Gamma_\tau$ is finite cyclic.
 
 \itemrm{2.} For $s\in\Qbb\cup\{\infty\}$, the isotropy group $\Gamma_s$ has finite index in the isotropy group $\SL_2(\Zbb)_s$.
\end{prop}

\begin{defi} \

 \itemrm{1.} Let $\tau\in\mc{H}$. The \emph{period} of $\tau$ is the number \[h_\tau=\abs{\!\pm\Gamma_\tau/\{\pm\id\}}\] of maps in the isotropy group of $\tau$. The period of $\tau$ only depends on the class $\Gamma\tau$. If $h_\tau > 1$, then we call the point $\tau$, or interchangeably the point $\pi(\tau)$, an \emph{elliptic point} for $\Gamma$. We further define the map $\delta_\tau\cl\mc{H}^*\to\Cbb$ to be the map represented by the matrix
 \[\delta_\tau = \begin{pmatrix}1 & -\tau \\ 1 & -\conj{\tau}\end{pmatrix}\in\GL_2(\Cbb)\] and the map $\rho_\tau\cl\Cbb\to\Cbb$ by $\rho_\tau(z)=z^{h_\tau}.$
 
 \itemrm{2.} Let $s\in\Qbb\cup\{\infty\}$. The \emph{width} of $s$ is the number \[h_s=[\SL_2(\Zbb)_s:\pm\Gamma_\tau].\] This also only depends on the class $\Gamma s$. We call the point $s$, or interchangeably the point $\pi(s)$, a \emph{cusp} for $\Gamma$. Furthermore, we define $\delta_s\in\SL_2(\Zbb)$ to be any map taking $s$ to $\infty$. Finally, define the map $\rho_s\cl\mc{H}^*\to\Cbb$ by $\rho_s(z)=\exp(2\pi iz/h_s)$. Note that this is well-defined on $\infty$ since we restrict to the upper half-plane.
\end{defi}

\begin{rmk}
 There are only two elliptic points on $X(\SL_2(\Zbb))$, namely the $\SL_2(\Zbb)$-classes of $i$ and $\mu_3$, where $\mu_3=\exp(2\pi i/3)$. Their periods are $2$ and $3$, respectively. Since $\Gamma_\tau\subseteq\SL_2(\Zbb)_\tau$, each elliptic point $\tau$ for $\Gamma$ has period $2$ or $3$, and lies in one of the classes $\Gamma i$ or $\Gamma \mu_3$, according to wether its period is $2$ or $3$. Since $\Gamma$ has finite index in $\SL_2(\Zbb)$, it follows that there are only finitely many elliptic points on $X(\Gamma)$.
\end{rmk}

\begin{rmk} \label{pr:width}
 The only cusp of $\SL_2(\Zbb)$ is $\infty$, whose isotropy subgroup is the group generated by $\smatrix{1&1\\0&1}$. Hence the width of $\infty$ with respect to $\Gamma$ is generally the smallest $h$ such that $\smatrix{1&h\\0&1}\in\Gamma$, \cf Remark \ref{pr:smallest-h}. The only exception is when $-\smatrix{1&h\\0&1}\in\Gamma_\infty$ but $\smatrix{1&h\\0&1}\notin\Gamma_\infty$, in which case if $h$ is the width, then $2h$ is minimal such that $\smatrix{1&2h\\0&1}\in\Gamma$. However, if $k$ is even then $h$ is still the period of a modular form of weight $k$. If $k$ is odd, then $h$ is only the skew period, while $2h$ is the period.
\end{rmk}

For $\tau\in\mc{H}^*$, let $U_\tau$ be an open neighborhood of $\tau$ such that for all $\gamma\in\mc{H}$, if $\gamma(U_\tau)$ meets $U_\tau$ then $\gamma\in\Gamma_\tau$, as per Corollary \ref{pr:chart-nhood}. If $\tau$ is an elliptic point or a cusp for $\Gamma$, we may assume that $U_\tau$ contains no further preimages of elliptic points or cusps. If $\pi(\tau)$ is neither, we may assume that $U_\tau$ contains no ellliptic points or cusps altogether.

Define the map $\varphi_\tau=\rho_\tau\circ\delta_\tau|_{U_\tau}$. This is a map with open image in $\Cbb$. For $\tau_1,\tau_2\in U_\tau$, we have $\varphi_\tau(\tau_1)=\varphi_\tau(\tau_2)$ if and only if $\pi(\tau_1)=\pi(\tau_2)$. Hence, the map $\varphi_\tau$ induces an injective continuous map $\psi_\tau\cl U_\tau\to\Cbb$. We take $(\pi(U_\tau),\psi_\tau)$ to be our chosen chart around $\pi(\tau)$.

\begin{prop} \label{pr:charts}
 The charts $(\pi(U_\tau),\psi_\tau)_{\tau\in\mc{H}^*}$ form a holomorphic atlas for the modular curve $X(\Gamma)$.
\end{prop}

\begin{prop}
 Let $\Gamma_1$ and $\Gamma_2$ be two conjugation subgroups with $\Gamma_1\subseteq\Gamma_2$. If $-\id\in\Gamma_2\ssm\Gamma_1$, then the induced morphism $X(\Gamma_1)\to X(\Gamma_2)$ has degree $[\Gamma_2:\Gamma_1]/2$, else it has degree $[\Gamma_2:\Gamma_1]$. The ramification index of a point $\pi_1(\tau)\in X(\Gamma_1)$ is $[\pm\Gamma_{2,\tau}:\pm\Gamma_{1,\tau}]$.
\end{prop}

\begin{cor}
 Let $f\cl X(\Gamma)\to X(\SL_2(\Zbb))$ be the morphism induced by the inclusion $\Gamma\in\SL_2(\Zbb)$. The ramification index of a point $\pi(\tau)\in X(\Gamma)$ is $h_\tau$, if $\tau$ is a cusp for $\Gamma$ or if $\tau$ is an elliptic point for $\SL_2(\Zbb)$ but not for $\Gamma$. Else, the ramification index is $1$.
\end{cor}

Let $y_2$, $y_3$, and $y_\infty$ be the images of $i$, $\mu_3$, and $\infty$, respectively, under the projection $\mc{H}^*\to\ X(\SL_2(\Zbb))$. For $h\in\{2,3\}$, let $\varepsilon_h$ denote the number of elliptic points in $X(\Gamma)$ of period $h$. Further, denote by $\varepsilon_\infty$ the number of cusps in $X(\Gamma)$.

The ramification locus of $f$ is contained in the set ${y_1,y_2,y_\infty}$. Now let $d$ be the degree of $f$. By applying the previous corollary and the formula 
\[d=\sum_{x\in f^{-1}(y_h)}e_f(x),\] one verifies the formulae
\[\sum_{x\in f^{-1}(y_h)}(e_f(x)-1)=\frac{h-1}{h}(d-\varepsilon_h),\]
\[\sum_{x\in f^{-1}(\infty)}(e_f(x)-1)=d-\varepsilon_\infty.\]

\begin{prop}
 The genus $g$ of the modular curve $X(\Gamma)$ is given by
 \[g=1+\frac{d}{12}-\frac{\varepsilon_2}{4}-\frac{\varepsilon_3}{3}-\frac{\varepsilon_\infty}{2}.\]
\end{prop}

\begin{proof}
 The statement follows from the Riemann--Hurwitz formula, the previous discussion, and the fact that $X(\SL_2(\Zbb))$ has genus $0$, which will be proved later in Example \ref{ex:j-invariant}.
\end{proof}

\begin{cor} \label{pr:inequality}
 We have the inequality \[2g-2+\frac{1}{2}\varepsilon_2+\frac{2}{3}\varepsilon_3+\varepsilon_\infty\geq0.\]
\end{cor}

\subsection{Automorphic forms and modular forms}

\begin{defi}
 Let $\gamma = \left(\begin{smallmatrix}a & b \\ c & d\end{smallmatrix}\right)$ be an element of $\GL_2(\Cbb)$ and let $f\cl\mc{H}\to\what{\Cbb}$ be a holomorphic function. For $\tau\in\mc{H}$ define the \emph{factor of automorphy}
 \[\mathrm{j}(\gamma,\tau)\defeq c\tau+d\]
 and for $k\in \Zbb$ the function $f[\gamma]_k\cl\mc{H}\to\Cbb$ by
 \[f[\gamma]_k(\tau) \defeq \det(\gamma)^{k/2}\jrm(\gamma,\tau)^{-k}f(\gamma\tau).\]
\end{defi}

\begin{remark}
 Let $\gamma,\gamma'\in\SL_2(\Zbb)$ and $\tau\in\mc{H}$.
 
 1. The factor of automorphy satisfies $\jrm(\gamma\gamma',\tau) = \jrm(\gamma,\gamma'(\tau))\jrm(\gamma',\tau)$.
 
 2. For all holomorphic functions $f:\mc{H}\to\Cbb$, we have $f[\gamma\gamma']_k=(f[\gamma]_k)[\gamma']_k$.
\end{remark}

\begin{defi}
 Let $f\cl \mc{H} \to \Cbb$ be a meromorphic function. Let $h\in\Nbb$ be minimal with the property that $\left(\begin{smallmatrix}1 & h\\0 & 1\end{smallmatrix}\right)\in\Gamma$. For $\tau\in\mc{H}$, set $q_h=\exp(2\pi i\tau/h)$.
 
  1. The function $f$ is \emph{$h\Zbb$-periodic}, if it satisfies $f(\tau+h)=f(\tau)$ for all $\tau\in\mc{H}$.

  2. Assume that the function $f$ is $h\Zbb$-periodic. If $f$ has no poles in the subset $\mc{N}_C$ for some $C>0$, then there exists a meromorphic function $\tilde f\cl B\setminus \{0\} \to \Cbb$ such that $f(\tau)=\tilde f(q_h)$ for all $\tau$. We say that $f$ is \emph{meromorphic at infinity,} if it has no poles in some $\mc{N}_C$ and if the associated function $\tilde f$ has a meromorphic continuation to the whole of $B$. In this case, we may write $f(\tau)=\sum_{n=m}^\infty a_n q_h^n$, with $m\in\Zbb$ and $a_m\neq 0$. We call $m$ the \emph{order of $f$ at infinity}, and we denote it by $\nu_\infty(f)$. The function $f$ is \emph{holomorphic at infinity} if $\nu_\infty(f)\geq 0$. We denote the order of $f$ at a point $\tau$ of $\mc{H}$ by $\nu_\tau(f)$.
  
  3. The function $f$ is said to satisfy the \emph{modular condition of weight $k$ with respect to $\Gamma$}, if $f[\gamma]_k=f$ for all $\gamma \in \Gamma$. Such a function is $h\Zbb$-periodic, since $\left(\begin{smallmatrix}1 & h\\0 & 1\end{smallmatrix}\right)\in\Gamma$.
  
  For all $\alpha\in\SL_2(\Zbb)$, the group $\alpha^{-1}\Gamma\alpha$ is a conjugation subgroup. Now assume that the function $f$ satisfies the modular condition of weight $k$ with respect to $\Gamma$. For all $\alpha$, the function $f[\alpha]_k$ satisfies the modular condition of the same weight with respect to the subgroup $\alpha^{-1}\Gamma\alpha$, and is hence $h_\alpha\Zbb$-periodic for some $h_\alpha$.
  
  4. Let $s\in\Qbb\cup\{\infty\}$ be a cusp for $\Gamma$. We define $f$ to be \emph{meromorphic at $s$}, if for some $\alpha\in\SL_2(\Zbb)$ with $\alpha (\infty)=s$, the function $f[\alpha]_k$ is meromorphic at infinity. The \emph{order} $\nu_s(f)$ of $f$ at $s$ is the order of the function $f[\alpha]_k$ at infinity. The function $f$ is \emph{holomorphic at $s$}, if $\nu_s(f)\geq 0$; it \emph{vanishes at $s$}, if $\nu_s(f)>0$. These notions do not depend on the choice of $\alpha$. 
  
  5. The function $f$ is an \emph{automorphic form} with respect to $\Gamma$, if it satisfies the modular condition of some weight $k$ with respect to $\Gamma$ and is meromorphic at all cusps of $\Gamma$. We call $k$ the \emph{weight} of $f$.
  
  6. An automorphic form $f$ is a \emph{modular form}, if it is holomorphic and is holomorphic at all cusps of $\Gamma$.
  
  7. A modular form $f$ is a \emph{cusp form}, if it vanishes at all cusps of $\Gamma$. 
\end{defi}

We denote the space of automorphic forms of weight $k$ by $\mathrm{A}_k(\Gamma)$, the space of modular forms of weight $k$ by $\Mrm_k(\Gamma)$, and the space of cusp forms of weight $k$ by $\mc{S}_k(\Gamma)$.

\begin{rmk}
 The $\Cbb$-algebra $\mathrm{A}_0(\Gamma)$ of automorphic forms of weight $0$ is isomorphic to the algebra $\Cbb(X(\Gamma))$ of meromorphic functions on the modular curve $X(\Gamma)$. To see this, note that the two notions of meromorphy at a cusp are equivalent. Similarly the space $\Mrm_0(\Gamma)$ is the space of holomorphic functions on $X(\Gamma)$, so $\Mrm_0(\Gamma)\simeq\Cbb$. Accordingly, $\mc{S}_0(\Gamma)=\{0\}$
\end{rmk}

\begin{expl} \label{ex:j-invariant}
 Let $\Delta$ and $G_2$ be the cusp form of weight $12$, respectively the modular form of weight $3$ of Example \ref{ex:delta}. The \emph{j-invariant} is the automorphic form of weight $0$ defined as $j=1728 G_2^3 / \Delta$. The function $\Delta$ has no zero on $\mc{H}$, but it is a cusp form with a nontrivial Fourier coefficient at place $1$ in its expansion at infinity. Hence the automorphic form $j$ has only one pole, which is simple. Therefore the j-invariant may be seen as a holomorphic function $j\cl X(\SL_2(\Zbb)) \to\what\Cbb$ of degree $1$. It follows that the modular curve $X(\SL_2(\Zbb))$ is isomorphic to the sphere $\what \Cbb$.
\end{expl}

\begin{rmk} \label{pr:nonzero-elt}
 The derivative $j'$ of the j-invariant is a nonzero element of $\mathrm{A}_2(\Gamma)$. Hence for $k$ even with $k\geq 2$, the algebra $\mathrm{A}_k(\Gamma)$ contains nonzero elements. Furthermore, if $f$ is a nonzero element of $\mathrm{A}_k(\Gamma)$, then division by $f$ yields an isomorphism \[\mathrm{A}_k(\Gamma)\simeq\mathrm{A}_0(\Gamma)f\simeq\Cbb(X(\Gamma))f.\]
\end{rmk}

Since the factor of holomorphy $\jrm(\gamma,\tau)$ has no zeroes or poles at non-cusp points, the order of an automorphic form at a point $\tau$ of $\mc{H^*}$ does not depend on the $\Gamma$-class of $\tau$. Hence we may define the order of an automorphic form at points of a modular curve, while taking the local coordinates into account.
\begin{defi}
 Let $f$ be an automorphic form of weight $k$; let $\tau\in\mc{H^*}$. The \emph{order} $\nu_{\pi(\tau)}(f)$ of $f$ at the point $\pi(\tau)$ of the modular curve $X(\Gamma)$ is defined as the number
 \[\nu_{\pi(\tau)}(f)=\begin{dcases*}\frac{\nu_{\tau}(f)}{h_\tau} & if $\tau\in\mc{H}$, \\
                                  \frac{\nu_{\tau}(f)h_\tau}{p_\alpha} & if $\tau$ is a cusp,\end{dcases*}\]
 where in the second case $\alpha$ is some element of $\SL_2(\Zbb)$ with $\alpha(\infty)=\tau$, and $p_\alpha$ is the period of the function $f[\alpha]_k$. The order at $\pi(\tau)$ does not depend on the choice of the representative $\tau$ or of $\alpha$. As seen in Remark \ref{pr:width}, the number $p_\alpha$ is in most cases equal to the width $h_\tau$, the exception takes place when $k$ is odd and $(\alpha\Gamma\alpha^{-1})_\infty=\gen{-\smatrix{1&h_\tau \\ 0&1}}$, in which case $p_\alpha=2h_\tau$.
\end{defi}

\begin{rmk}
 This definition does what one expects when looking at the Laurent expansion of the function induced by $f$ on the small neighborhoods $\pi(U)$ as in Proposition \ref{pr:charts}. If $\tau$ is a non-cusp, then writing $f|_U=f_{\text{local}}\circ\rho\circ\delta$ with $f_{\text{local}}\cl\varphi(U)\to V$ shows that the order $\nu_\tau(f)$ is just the order of $f_{\text{local}}$ at $\pi(\tau)$. If $\tau$ is a cusp, the extra factor accounts for possible discrepancies between the width of $\tau$ and the modified function $f[\alpha]_k$. In particular, the order function $\nu_{\pi(\tau)}$ satisfies the familiar rules \[\nu_{\pi(\tau)}(f_1+f_2)\geq\min\{\nu_{\pi(\tau)}(f_1),\nu_{\pi(\tau)}(f_2)\}\] and \[\nu_{\pi(\tau)}(f_1f_2)=\nu_{\pi(\tau)}(f_1) + \nu_{\pi(\tau)}(f_2).\]
\end{rmk}

\subsection{The dimension formula}

\begin{defi}
 Let $Y$ be a Riemann surface. Let $\Trm^*\!(Y)$ denote the cotangent bundle of $Y$. For $n\in\Nbb$, define the sheaf of \emph{$n$-fold meromorphic differentials} $\Omega(Y)^{\otimes n}$ to be the sheaf of sections into the bundle $\Trm^*\!(Y)^{\otimes n}$.
\end{defi}

The natural morphism $\pi\cl\mc{H}\to X(\Gamma)$ gives rise to the pullback map \[\pi^*\cl\Omega(X(\Gamma))^{\otimes n}\to\Omega(\mc{H})^{\otimes n}.\]
Note that for $\delta\in\GL_2(\Cbb)$ we have $\delta'=\det(\delta)\jrm(\delta,\tau)^{-2}$. Hence for $f\in\Omega(\mc{H})^{\otimes n}$ we have the relation $f[\alpha]_{2n} = \alpha^* f$.
If $U$ is the domain of some chart for some Riemann surface, then the $n$-fold meromorphic differentials are naturally identified with the meromorphic functions on $U$. In the following, we shall make implicit use of this fact.

\begin{rmk} \label{pr:morse-sard}
 Let $\phi\cl Y_1\to Y_2$ be a morphism of Riemann surfaces with dense image. Then the pullback map $\phi^*\cl\Omega(Y_2)^{\otimes n}\to\Omega(Y_1)^{\otimes n}$ is injective by the Morse-Sard theorem.
\end{rmk}

\begin{prop} \label{pr:gluing-differentials}
 Let $\phi\cl Y_1\to Y_2$ be a morphism of Riemann surfaces. Let $f$ be a meromorphic function on $Y_1$. For each chart domain $V$ in $Y_2$, let a holomorphic function $\omega_V\cl V\to \Cbb$ be given such that $\phi^*\omega_V=f|_{\phi^{-1}(V)}$. Then the collection $\{\omega_V\}_V$ forms an $n$-fold differential on $Y_2$.
\end{prop}

\begin{proof}
 Let $V'$ and $V$ be two chart domains in $Y_2$, let $U$ be their intersection. Set $W=\phi^{-1}(U)$. Since the map $\phi|_W(U)\cl W\to U$ is surjective, by the previous remark it suffices to show that $(\phi|_W)^*\omega_{V'}=(\phi|_W)^*(g_{V',V}\ \omega_V)$, where $g_{V',V}$ is the corresponding cocycle in the cotangent bundle. But a calculation shows that $(\phi|_W)^*(g_{V',V}\omega_V)=(\phi|_W)^* \omega_V$, and by assumption $(\phi|_W)^* \omega_V' = (\phi|_W)^* \omega_V$.
\end{proof}

\begin{prop}

 Let $\omega\in\Omega(X(\Gamma))^{\otimes n}$, write $\pi^*(\omega)=f$ for some meromorphic function $f\cl \mc{H}\to\what\Cbb$. Then $f$ is an automorphic form of weight $2n$.
\end{prop}

\begin{proof}
 Let $\gamma\in\Gamma$. Then $\gamma^* \pi^* \omega = \pi^* \omega$, since $\pi \gamma = \gamma$. On the other hand, $\gamma^* \pi^* \omega = (\gamma')^n(f\circ\gamma)$. Hence we have $f=f[\gamma]_{2n}$, so that the $f$ satisfies the modular condition. Furthermore, $f$ is meromorphic at each cusp because $\omega$ is.
\end{proof}

By this proposition we get a map of $\Cbb$-vector spaces \[\pi^*\cl\Omega(X(\Gamma))^{\otimes n}\to\mathrm{A}_{2n}(\Gamma).\]

\begin{prop} \label{pr:bijection}
 The map $\pi^*$ is bijective. If $\omega\in\Omega(X(\Gamma))^{\otimes n}$ and $f=\pi^*(\omega)$, then the orders of $f$ and $\omega$ at a point $\pi(\tau)\in X(\Gamma)$ are related by the formula
 \[\nu_{\pi(\tau)}(\omega)=\begin{dcases*}\nu_{\pi(\tau)}(f)-n\left(1-\frac{1}{h}\right) & if $\tau\in\mc{H}$ and $\tau$ has period $h$, \\
 \nu_{\pi(\tau)}(f)-n & if $\tau\in\Qbb\cup\{\infty\}$\end{dcases*}.\]
\end{prop}

\begin{proof}
 The map $\pi^*$ is injective by Remark \ref{pr:morse-sard}, since the map $\pi$ has dense image.

 For surjectivity, let $f$ be an automorphic form of weight $2n$, viewed as an element of $\Omega(\mc{H})^{\otimes n}$ when convenient. Let $\tau_0\in\mc{H}^*$, and $U\defeq U_{\tau_0}$ be the corresponding open set as in \ref{pr:charts}, such that there is a chart of $X(\Gamma)$ of the form $\psi\cl \pi(U)\stackrel{\sim}{\longrightarrow} V$. For convenience, we are omitting the index $\tau_0$. Hence we have a diagram
 \[
 \begin{tikzcd}
  U \arrow{r}{\pi} \arrow{d}{\delta}[swap]{\rotatebox{90}{\(\simeq\)}} &
  \pi(U) \arrow{d}{\psi}[swap]{\rotatebox{90}{\(\simeq\)}} \\
  \delta(U) \arrow{r}{\rho} &
  V\rlap{,}
 \end{tikzcd}
 \]
 with $\delta$ and $\rho$ as in the discussion preceding Proposition \ref{pr:charts}. Define the $n$-fold differential $\lambda=(\delta^{-1})^*(f|_{U\cap\mc{H}})$ on $\delta(U)$, which we also view as a meromorphic function. We distinguish between two cases, depending on whether $\tau_0$ is a cusp or not.
 
 Suppose $\tau_0$ is not a cusp and has period $h$. The extended isotropy group $\pm(\delta\Gamma\delta^{-1})_0/\{\pm\id\}$ is generated by the rotation $r_h$ represented by the matrix $\smatrix{\mu_{2h}&0\\0&\mu_{2h}^{h-1}}$, where $\mu_{2h}=\exp(2\pi i/h)$. Since the function $f$ is $\Gamma$-invariant (under pullbacks), the function $\lambda$ is $\delta\Gamma\delta^{-1}$-invariant; in particular, for $z\in\delta(U)$ we have $\lambda(r_h z)=\jrm(r_h,z)^{2n}\lambda(z)$, from which the relation $(\mu_h z)^n\lambda(\mu_h z)=z^n \lambda(z)$ follows. Hence the function $\zeta\mapsto z^n\lambda(z)$ is invariant under the transformation $z\mapsto\mu_h z$, so that there is a meromorphic function $g$ on $\delta(U)$ such that $g(z^h)=z^n\lambda(z)$ for all $z$. We have $h\nu_0(g)=n+\nu_{\tau_0}(f)$, hence $\nu_0(g)=\nu_{\pi(\tau_0)}(f)+n/h$.
 
 We define the $n$-fold differential $\theta$ on $V$ by putting $\theta(z)=g(z)/(hz)^n$, and set $\omega_{\pi(U)}=\psi^*\theta$. A calculation shows that $\rho^*\theta=\lambda$, therefore we have $\pi^*\omega_{\pi(U)}=f|_U$.
 
 The case where $\tau_0$ is a cusp is treated analogously: here, $V$ is the open unit ball, now the extended isotropy group $\pm(\delta\Gamma\delta^{-1})_\infty/\{\pm\id\}$ is generated by the matrix $\smatrix{1&h\\0&1}$, where $h$ is the width of $\tau_0$, and the function $\lambda$ is $h\Zbb$-periodic. Hence there is a meromorphic function $g$ on $V\ssm\{0\}$ such that $g(e^{2\pi i z / h})=\lambda(z)$ for all $z$. Since $f$ is holomorphic at the cusp $\tau_0$, the function $g$ has a meromorphic continuation on all of $V$. Its order at $0$ is given by $\nu_0(g)=\nu_{\pi(\tau_0)}(f)$.
 
 Define the $n$-fold differential $\theta$ on $V$ by putting $\theta(z)=h^ng(z)/(2\pi i z)^n$, and set again $\omega_{\pi(U)}=\psi^*\theta$, an $n$-fold differential on $\pi(U)$. Another calculation shows that $\rho^*\theta=\lambda$, hence $\pi^*\omega_{\pi(U)}=f|_{U\cap \mc{H}}$.
 
 Now we put everything together: by Proposition \ref{pr:gluing-differentials}, the collection $\{\omega_{\pi(U)}\}_U$ forms an $n$-fold differential on $X(\Gamma)$, as required.
 
 By construction, we obtain the orders of the differential as in the statement.
\end{proof}

We now come to the Riemann–Roch theorem, stated here without proof, and its application to computing the dimension of $\Mrm_k(\Gamma)$ for $k$ even.
Let $X$ be a compact Riemann surface. To a divisor $D\in\Div(X)$ we associate its \emph{linear space} $L(D)$, a finite-dimensional vector space over $\Cbb$ whose dimension we denote by $\ell(D)$, defined as
\[L(D)=\{f\in\Cbb(X)\scl \divi(f)+D\geq 0\}\cup\{0\}.\]
\begin{thm}[Riemann–Roch]
 Let $X$ be a compact Riemann surface, $g$ its genus. Let $K$ be a canonical divisor on $X$. If $D\in\Div^0(X)$, then
 \[\ell(D)=\deg(D)-g+1+\ell(K-D).\]
\end{thm}

\begin{rmk} \label{pr:rrcor} From the Riemann–Roch theorem follow:
 
 1. We have $\ell(K) = g$ and $\deg(K) = 2g-2$.

 2. If $\deg(D)<0$, then $\ell(D)=0$.
 
 3. If $\deg(D)>2g-2$, then $\ell(D)=\deg(D)-g+1$.
\end{rmk}

\begin{defi}
 Let $X$ be a Riemann surface, $k$ an even integer.
 
 1. Define the group $\Div_\Qbb(X)$ of \emph{rational divisors} on $X$ by $\Div_\Qbb(X)=\Div(X)\otimes\Qbb$. This group is equipped with a relation $\leq$, which extends the relation on $\Div(X)$.
 
 2. Let $f\in\mathrm{A}_k(\Gamma)$ be a nonzero automorphic form. The \emph{rational divisor} $\divi(f)$ of $f$ is the rational divisor $\divi(f)=\sum_{x\in X(\Gamma)}\nu_x(f)x$.
 
 3. The floor function $\floor{\ }\cl\Div_\Qbb(X)\to\Div(X)$ is defined by applying the floor function coefficient-wise.
\end{defi}

\begin{rmk}
 Any differential in $\Omega(X(\Gamma))^{\otimes n}$ has degree $2n(g-1)$. To see this, choose a nonzero element $f$ of $\mathrm{A}_2(\Gamma)$, and let $\lambda\in\Omega(X(\Gamma))^{\otimes 1}$ be the preimage of $f$ under $\pi^*$. The divisor $\divi(\lambda)$ is a canonical divisor and has hence degree $2(g-1)$ by the previous remark. The element $\lambda^n$ of $\Omega(X(\Gamma))^{\otimes n}$ has degree $2n(g-1)$. Since we may write $\Omega(X(\Gamma))^{\otimes n}=\Cbb(X(\Gamma))\lambda^{n}$, it follows that all elements of $\Omega(X(\Gamma))^{\otimes n}$ have the same degree as $\lambda^{n}$.
\end{rmk}

\begin{thm}
 Let $k$ be an even integer, set $n=k/2$. Let $g$ be the genus of the modular curve $X(\Gamma)$; for $h\in\{2,3\}$, let $\varepsilon_h$ denote the number of elliptic points in $X(\Gamma)$ of period $h$. Let $\varepsilon_\infty$ be the number of cusps. The following formula for the dimension of the space $\Mrm_k(\Gamma)$ of modular forms with respect to the congruence subgroup $\Gamma$ holds:
 \[
  \dim(\Mrm_k(\Gamma))=\begin{dcases}
                      (k-1)(g-1)+\floor{\frac{k}{4}}\varepsilon_2+\floor{\frac{k}{3}}\varepsilon_3+\frac{k}{2}\,\varepsilon_\infty & \text{ if $k\geq 2$ ,}\\
                      1 & \text{ if $k=0$,}\\
                      0 & \text{ if $k\leq 0$}
                     \end{dcases}.
 \]
\end{thm}
\begin{proof}
 First let $k\leq 0$. We have $\Mrm_0(\Gamma)\simeq\Cbb$ and $\mc{S}_0(\Gamma)=\{0\}$. If $k<0$ and $f\in\Mrm_{k}(\Gamma)$, then $f^{12}\Delta^{-k}$ lies in $\mc{S}_0(\Gamma)$, where $\Delta$ is the usual weight $12$ cusp form. Hence $f=0$.
 
 Now let $k\geq 2$. By Remark \ref{pr:nonzero-elt}, there is a nonzero element $f$ of $\mathrm{A}_k(\Gamma)$. Let $\omega\in\Omega(X(\Gamma))^{\otimes n}$ be the preimage of $f$ under $\pi^*$. Using the isomorphism $\mathrm{A}_k\simeq\Cbb(X(\Gamma))f$, we obtain an isomorphism
 \[
 \Mrm_k(\Gamma)\simeq\{f_0\in\Cbb(X(\Gamma))\scl \divi(f_0)+\divi(f)\geq 0\} \cup \{0\}.
 \]
 Since $\divi(f_0)$ is integral, the above condition is equivalent to $\divi(f_0)+\floor{\divi(f)}\geq 0$, hence there is an isomorphism $\Mrm_{k}(\Gamma)\simeq L(\floor{\divi(f)})$, and $\dim(\Mrm_k(\Gamma))=\ell(\floor{\divi(f)})$.
 
 Using the formula for the orders found in Proposition \ref{pr:bijection}, we obtain
 \[
  \divi(\omega)=\floor{\divi(f)}-\left(\sum_{x_2}\floor{\frac{k}{4}} x_{2}+\sum_{x_{3}}\floor{\frac{k}{3}} x_3+\sum_{x_{\infty}}\frac{k}{2}x_{\infty}\right).
 \]
 Here the variables $x_2, x_3$ and $x_\infty$ run through the elliptic points of period $2$, the elliptic points of period $3$, and the cusps, respectively; all seen as points of $X(\Gamma)$. It follows with the previous remark that
 \[
  \deg(\floor{\divi(f)})=k(g-1) + \floor{\frac{k}{4}}\varepsilon_2 + \floor{\frac{k}{3}}\varepsilon_3 + \frac{k}{2}\varepsilon_\infty.
 \]
 We have $\floor{k/4}\geq(k-2)/4$ and $\floor{k/3}\geq(k-2)/3$. Together with Corollary \ref{pr:inequality}, this implies
 \[
  \deg(\floor{\divi(f)})>2g-2.
 \]
 Therefore, by Corollary \ref{pr:rrcor},
 \[
  \ell(\floor{\divi(f)})=(k-1)(g-1) + \floor{\frac{k}{4}}\varepsilon_2 + \floor{\frac{k}{3}}\varepsilon_3 + \frac{k}{2}\varepsilon_\infty.
 \]
 
\end{proof}

