\section{Modular curves}

\newcommand{\jrm}{\mathrm{j}}

One may weaken the definition of a modular form by requiring that modular condition be met only for transformations lying in certain subgroups $\Gamma$ of $\SL_2(\Zbb)$. In this section we will calculate the dimension of the associated space of weight $k$ modular forms $\M_k(\Gamma)$ using the fact that modular forms can be seen as section of a certain line bundle on a special Riemann surface, the modular curve associated to the subgroup $\Gamma$.

\subsection{Congruence subgroups and modular forms}

\begin{defi}
 Let $N\in \Zbb$.
 
 1. The \emph{principal congruence subgroup} of \emph{level} $N$ is the subgroup
 \[\Gamma(N)=\left\{\begin{pmatrix} a & b \\ c & d \end{pmatrix}\in\SL_2(\Zbb)\scl \begin{pmatrix}a & b \\ c & d\end{pmatrix} \equiv \begin{pmatrix}1 & 0 \\ 0 & 1\end{pmatrix} \pmod{N}\right\}.\]
 
 2. A \emph{congruence subgroup} is a subgroup $\Gamma\subset\SL_2(\Zbb)$ such that $\Gamma(N)\subset\Gamma$ for some $N\in\Zbb$. We then say that $\Gamma$ has \emph{level} $N$. 
\end{defi}

\begin{rmk}
 The subgroup $\Gamma(N)$ is the kernel of the component-wise congruence map $\SL_2(\Zbb)\to\SL_2(\Zbb/N\Zbb)$. It is hence normal in $\SL_2(\Zbb)$ and of finite index. Consequently, each congruence subgroup has finite index in $\SL_2(\Zbb)$, while not being necessarily normal.
\end{rmk}

\begin{defi}
 Let $\gamma = \left(\begin{smallmatrix}a & b \\ c & d\end{smallmatrix}\right)$ be an element of $\GL_2(\Cbb)$ and let $f\cl\mc{H}\to\what{\Cbb}$ be a holomorphic function. For $\tau\in\mc{H}$ define the \emph{factor of automorphy}
 \[\mathrm{j}(\gamma,\tau)\defeq c\tau+d\]
 and for $k\in \Zbb$ the function $f[\gamma]_k\cl\mc{H}\to\Cbb$ by
 \[f[\gamma]_k(\tau) \defeq \det(\gamma)^{k/2}\jrm(\gamma,\tau)^{-k}f(\gamma\tau).\]
\end{defi}

\begin{remark}
 Let $\gamma,\gamma'\in\SL_2(\Zbb)$ and $\tau\in\mc{H}$.
 
 1. The factor of automorphy satisfies $\jrm(\gamma\gamma',\tau) = \jrm(\gamma,\gamma'(\tau))\jrm(\gamma',\tau)$.
 
 2. For all holomorphic functions $f:\mc{H}\to\Cbb$, we have $f[\gamma\gamma']_k=(f[\gamma]_k)[\gamma']_k$.
\end{remark}

\begin{defi} Let $f\cl \mc{H} \to \Cbb$ be a function, let $\Gamma$ be a congruence subgroup. Hence $\Gamma$ contains an element of the form $\left(\begin{smallmatrix}1 & *\\0 & 1\end{smallmatrix}\right)\in\Gamma$. Let $h\in\Nbb$ be minimal with the property that $\left(\begin{smallmatrix}1 & h\\0 & 1\end{smallmatrix}\right)\in\Gamma$. For $\tau\in\mc{H}$, set $q_h=\exp(2\pi i\tau/h)$.

  1. The function $f$ is \emph{$h\Zbb$-periodic}, if it satisfies $f(\tau + h) = f(\tau)$ for all $\tau \in \mc{H}$. Analogously to the case $h=1$, there exists a function $\tilde f\cl B\setminus \{0\} \to \Cbb$ such that $f(\tau)=\tilde f(q_h)$ for all $\tau$. Now let $f$ be holomorphic, so that $\tilde f$ is also holomorphic. We say that $f$ is \emph{holomorphic at infinity}, if $\tilde f$ has a holomorphic continuation to the whole of $B$.
  
  2. The function $f$ is said to satisfy the \emph{modular condition of weight $k$ with respect to $\Gamma$}, if $f[\gamma]_k=f$ for all $\gamma \in \Gamma$. Such a function is $h\Zbb$-periodic, since $\left(\begin{smallmatrix}1 & h\\0 & 1\end{smallmatrix}\right)\in\Gamma$.
  
  3. For all $\alpha\in\SL_2(\Zbb)$, the group $\alpha^{-1}\Gamma\alpha$ is a conjugation subgroup. Now assume that the function $f$ is holomorphic and that it satisfies the modular condition of weight $k$ with respect to $\Gamma$. Hence, for all $\alpha$, the function $f[\alpha]_k$ satisfies the modular condition of the same weight with respect to $\alpha^{-1}\Gamma\alpha$, and is hence $h_\alpha\Zbb$-periodic for some $h_\alpha$. We define $f$ to be \emph{holomorphic at all cusps of $\Gamma$}, if for all $\alpha\in\SL_2(\Zbb)$, the function $f[\alpha]_k$ is holomorphic at $\infty$.
  
  4. The function $f$ is a \emph{modular form (of weight $k$) with respect to $\Gamma$} if it is holomorphic, satisfies the modular condition and is holomorphic at all cusps of $\Gamma$.
  
  5. The function $f$ is a \emph{cusp form of weight $k$ with respect to $\Gamma$}, if it is a modular form of weight $k$, and if the associated holomorphic function $\tilde{f}$ satisfies $\tilde{f}(0)=0$ after its holomorphic continuation. 
\end{defi}

\subsection{Modular curves}

Let $\Gamma$ be a congruence subgroup.

\begin{defi} \ 

  1. For $M>0$, define the set $\mc{N}_M \subset \Cbb\cup\{\infty\}$ by
  \[\mc{N}_M=\left\{\tau\in\mc{H}\scl \Img(\tau)>M\right\}\cup\{\infty\}.\]

  2. Define the \emph{compact upper half-plane} $\mc{H}^*$ to be the set
  \[\mc{H}^*=\mc{H}\cup\Qbb\cup\{\infty\},\]
  endowed with the topology generated by union of the topology of $\mc{H}$ with the set
  \[\left\{\alpha(\mc{N_M})\scl \alpha\in\SL_2(\Qbb), M\in\Rbb_{>0}\right\}.\]
  
  3. The \emph{modular curve} $X(\Gamma)$ is defined as the quotient space $_\Gamma\backslash\mc{H}^*$.
\end{defi}

For $\tau\in\mc{H}^*$, denote the stabilizer of $\tau$ under the action of $\Gamma$ by $\Gamma_\tau$.

\begin{prop}
 The group action of $\SL_2(\Zbb)$ on $\mc{H}^*$ is properly discontinuous. That is, if $\tau_1, \tau_2\in\mc{H}^*$, then there are open neighborhoods $U'_1$ of $\tau_1$ and $U'_2$ of $\tau_2$, such that for all but finitely many $\gamma\in\SL_2(\Zbb)$, the sets $\gamma(U'_1)$ and $U'_2$ do not meet.
\end{prop}

\begin{proof}
 We consider three cases separately.
 
 \itemrm{1.} Let $\tau_1,\tau_2\in\mc{H}$. Choose $U'_1$ and $U'_2$ to be any open neighborhoods with closure
 
 \itemrm{2.}
 
 \itemrm{3.}
\end{proof}

\begin{cor}
 Let $\tau_1$ and $\tau_2$ be elements of $\mc{H}^*.$ Then there exist open neighborhoods $U_1$ of $\tau_1$ and $U_2$ of $\tau_2$ such that for all $\gamma\in\SL_2(\Zbb)$, if $\gamma(U_1)$ meets $U_2$ then $\gamma\tau_1 = \tau_2$.
\end{cor}

\begin{proof}
 Let $U'_1$ and $U'_2$ be open neighborhoods of $\tau_1$ and $\tau_2$, respectively, such that the set
 \[\{\gamma\in\SL_2(\Zbb)\scl \gamma(U'_1)\cap U'_2\neq\emptyset, \gamma(\tau_1)\neq \tau_2\}\]
 is finite. We denote this set by $F$. For each $\gamma\in F$, choose disjoint open neighborhoods $U_{1,\gamma}$ and $U_{2,\gamma}$ of $\gamma\tau_1$ and $\tau_2$, respectively, and put
 \[U_1 = U'_1\cap\left(\bigcap_{\gamma\in F}\gamma^{-1}(U_{1,\gamma})\right)\text{ and}\]
 \[U_2 = U'_2\cap\left(\bigcap_{\gamma\in F}U_{2,\gamma}\right).\]
 Then $U_1$ and $U_2$ satisfy the required propertieties.
\end{proof}

\begin{cor} \ 

 \begin{rm}1\end{rm}. The modular curve $X(\Gamma)$ is Hausdorff.
 
 \begin{rm}2\end{rm}. Let $\tau\in\mc{H}^*.$ Then there is an open neighborhood $U$ of $\tau$ such that for all $\gamma\in\mc{H}$, if $\gamma(U)$ meets $U$ then $\gamma\in\Gamma_\tau$
\end{cor}







